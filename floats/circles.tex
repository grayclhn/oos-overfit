% TikZ commands to draw circles
\newcommand{\circlefigA}[4]{
  \begin{tikzpicture}
    \clip (-#3,-#3) rectangle (#4,#4);
    \fill[lightgray] (-#3,-#3) rectangle (#4,#4);
    \foreach \r in {1.089629,1.190782,1.310824,1.466089,1.709963,2.71521}
      \draw [black,very thin] (1,1) circle ({veclen(\r,\r)});
    % rejection region for F-test
    \filldraw[fill=white,draw=black] (0,0) circle (#1);
    \begin{scope}
      \fill[darkgray] (0,0) circle (#1);
      \clip (0,0) circle (#1);
    \foreach \r in {1.089629,1.190782,1.310824,1.466089,1.709963,2.71521}
        \draw [black,very thin] (1,1) circle ({veclen(\r,\r)});
    \end{scope}
    % circle of equal generalization error
    \filldraw[fill=white,draw=black,thick] (1,1) let \p1=(1,1) in circle({veclen(\x1,\y1)});
    \draw (1,1) let \p1=(1,1) in circle({veclen(\x1,\y1)});
    \fill [black] (1,1) circle (2pt) node[right] {$M_2 \theta_2$};
    \draw[thick] (0,0) circle (#1);
    \draw (1,1)--(0,0);
    % axes
    \draw[->] (0,0)--(#2,0) node[right,fill=lightgray] {$e_1 M_2 \bh{2T}$};
    \draw[->] (0,0)--(0,#2) node[above,fill=lightgray] {$e_2 M_2 \bh{2T}$};
  \end{tikzpicture}
}

% TikZ commands to draw circles
\newcommand{\circlefigB}[4]{
  \begin{tikzpicture}
    \clip (-#3,-#3) rectangle (#4,#4);
    \fill[white] (-#3,-#3) rectangle (#4,#4);
    \draw[thick] (0,0) circle (#1);
    \filldraw[fill=lightgray,thick] (0.75,0.75) let \p1=(0.75,0.75) in circle({veclen(\x1,\y1)});
    \draw[thick] (0,0) circle (#1);
    \fill [black] (.75,.75) circle (2pt) node[right,fill=white] {$M_2 \theta_2$};
      \foreach \r in {0.9177846,0.8403243,0.765711,0.6923306,0.6185878,0.5427191,0.4621276,0.3723508,0.2625747,0.0617173,0.03120708}
        \draw [black,very thin] (0.75,0.75) circle (\r);
    \begin{scope}
      \clip (0.75,0.75) let \p1=(0.75,0.75) in circle({veclen(\x1,\y1)});
      \fill [black] (.75,.75) circle (2pt) node[right,fill=lightgray] {$M_2 \theta_2$};
      \filldraw[fill=darkgray,draw=black] (0,0) circle (#1);
      \clip (0,0) circle (#1);
      \foreach \r in {0.9177846,0.8403243,0.765711,0.6923306,0.6185878,0.5427191,0.4621276,0.3723508,0.2625747,0.0617173,0.03120708}
        \draw [black,very thin] (0.75,0.75) circle (\r);
      \draw[thick] (0.75,0.75) let \p1=(0.75,0.75) in circle({veclen(\x1,\y1)});
    \end{scope}
    \fill [black] (.75,.75) circle (2pt);
    \draw[->] (0,0)--(#2,0) node[right] {$e_1 M_2 \bh{2T}$};
    \draw[->] (0,0)--(0,#2) node[above] {$e_2 M_2 \bh{2T}$};
  \end{tikzpicture}
}

%%% Local Variables:
%%% mode: latex
%%% TeX-master: "../paper"
%%% End:
