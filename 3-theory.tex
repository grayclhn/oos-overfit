\section{Theoretical results}
\label{sec:theory}

This section lays out our theoretical results. The first subsection
presents results for the DMW \oost\ test; we show that it is
asymptotically normal when one or both forecasting models is overfit.
(i.e., under our increasing-$K$ asymptotic approximation.) We also
present a new limitation on these statistics---OOS comparisons heavily
penalize overfit models unless the size of the test sample is a very
small proportion of the total sample size, which will not be practical
in much applied research. The second subsection presents results for
full-sample statistics and shows that widely used test statistics and
model selection criteria are misleading when choosing a model for
forecasting. In contrast to OOS comparisons, these full-sample
criteria choose overfit models too often, even when they are less
accurate than simple benchmark models. These results are somewhat
abstract, so the third subsection works through an example DGP in
detail.

\subsection{Asymptotic normality of the DMW test}
\label{sec:oostheory}

This section has two main conceptual results. The first,
Lemma~\ref{res-mixingale}, shows that the OOS average, $\oosA$, is
asymptotically normal as the size of the test sample grows, even when
the models are overfit. But $\oosA$ is centered at $\E_R \oosA$, the
difference in the generalization error of the models estimated over
the \emph{training sample}, which is not the quantity of interest to
forecasters. Forecasters will generally want to estimate or test
hypotheses about the difference in the generalization error of the
models estimated with the full sample, $\E_T \oosB$. Our second
result, Lemma~\ref{res-convergence}, shows that these quantities are
approximately equal only when the test sample is very small relative
to the total sample size. In particular, we show that $\oosA$ is a
consistent estimator of $\E_T \oosB$ when $P/T \to 0$ and is
asymptotically normal with mean $\E_T \oosB$ when $P^2/T \to 0$. After
establishing these Lemmas, we then show that the \oost\ test is
asymptotically standard normal and can be used to test hypotheses
about $\E_T \oosB$, which requires the two Lemmas as well as a
consistent estimator of the variance of the OOS average.

In the first result, we show that
$\sqrt{P} (\oosA - \E_R \oosA)$ is asymptotically normal as $P \to
\infty$. This application of the CLT is complicated by a hidden source
of dependence---the training sample estimators $\bh{iR}$. Under
conventional asymptotic theory, we would replace each $\bh{iR}$ with
its pseudotrue value $\theta_i$ and apply the CLT to
$L(y_{t+\h} - x_{it}'\theta_i)$, with some complications potentially
arising from the replacement (as in \citealp{Wes:96},
\citealp{ClM:01}, or \citealp{Mcc:07}, for example). But we can not
make that replacement here, because our asymptotic approximation
prevents $x_{it}'\bh{iR}$ from converging to $x_{it}'\theta_i$.

Instead, we show that $D_t - \E_R D_t$ is an $L_2$-mixingale that
satisfies the CLT. Mixingales satisfy a weak-dependence condition
similar to MDSes,\footnote{%
  An array $Z_{n,t}$ and an increasing sequence of \sfields
  $\Gs_{n,t}$ is an \emph{$L_2$-mixingale of size $-1/2$} if there is
  an array of constants $\{c_{n,t}\}$ and a sequence of constants
  $\zeta_l = O(l^{-1/2 - \delta})$ for some $\delta > 0$ such that
  \begin{equation*}
    \lVert \E(Z_{n,t} \mid \Gs_{n,t-l}) \rVert_2
    \leq c_{n,t} \zeta_l
    \qquad\text{and}\qquad
    \lVert Z_{n,t} - \E(Z_{n,t} \mid \Gs_{n,t+l}) \rVert_2
    \leq c_{n,t} \zeta_{l+1}.
  \end{equation*}
  Mixingales were introduced and developed by
  \citet{Mcl:74,Mcl:75,Mcl:75b,Mcl:77}.} %
but they have some limitations. Transformations of mixingales are
typically not themselves mixingales, which means that CLTs for
mixingale processes require additional assumptions to hold beyond the
mixingale property.\footnote{%
  See \citet{Jon:97} for an illustration. We will borrow heavily from
  his results in our proofs.} %
This is in contrast to Near Epoch Dependent (NED) processes, which
retain the NED property after transformations. \citep[See chapter 17
of][for further discussion of these properties.]{Dav:94} But the OOS
loss has more structure than most mixingales and behaves like an NED
process in important respects. (The key result here is
Lemma~\ref{lem:a2} in the Appendix.) Lemma~\ref{res-mixingale} presents the
CLT that we use later, and Section~\ref{sec:example} works through an
i.i.d. example in detail. For i.i.d. observations, the OOS loss is an
MDS and is easier to work with.

\begin{lem}\label{res-mixingale}
  If Assumptions~\ref{asmp-1}--\ref{asmp-3} hold then $\{D_t - \E_R
  D_t, \mathcal F_t\}$ is an $L_2$-mixingale of size $-1/2$.
  Moreover
  \begin{equation}
    \sqrt{P}(\oosA - \E_R \oosA)/\sigma \to^d N(0,1)
  \end{equation}
  as $P \to \infty$ if $\sigma^2$ is uniformly almost surely positive,
  where $\sigma^2 = \var_R(\sqrt{P} \oosA)$ (which is equal to $P
  \E_R(\bar{D}_R - \E_R \bar{D}_R)^2$).
\end{lem}

It may be helpful to compare Lemma~\ref{res-mixingale} to the method
of proof in \citet{GiW:06}. \citet{GiW:06} show that the OOS average
is asymptotically normal when the forecasts are estimated with a fixed
length rolling window. In that case, each $\bh{it}$ depends on only
the most recent $R$ observations and, since $R$ is fixed in their
theory, the forecast errors $y_{t+\h} - x_{it}'\bh{it}$ are themselves
mixing processes. Transformations of their forecast errors are
still obviously mixing processes and obey the CLT.

In our paper, $R$ is not fixed and the forecast errors are not a
convenient weakly-dependent process, since the estimation error in
$\bh{it}$ introduces strong dependence. Consequently,
transformations of the forecast errors are not weakly dependent
either. But this additional dependence has a special form and can be
removed by subtracting the conditional mean; specifically $g(y_{t+\h} -
x_{it}'\bh{it})$ is not weakly dependent but $g(y_{t+\h} -
x_{it}'\bh{it}) - \E_R g(y_{t+\h} - x_{it}'\bh{it})$ is.
Assumptions~\ref{asmp-1}--\ref{asmp-3} allow us to show directly
that $D_t - \E_R D_t$ and (crucially) $D^2_t - \E_R D_t^2$ are both
weakly dependent mixingales, ensuring Lemma~\ref{res-mixingale}.

The next Lemma connects $E_R \oosA$ to $\E_T \oosB$, the difference in
the models' generalization error. Under conventional asymptotics,
these quantities are generally close. But they are not for overfit
models and the models will tend to forecast better when estimated over the
full sample than over the training sample. Consequently, OOS
comparisons will penalize overfit models too much unless the test
sample is small relative to the total data set.

\begin{lem} \label{res-convergence}
  Under Assumptions \ref{asmp-1}--\ref{asmp-4}a,
  \begin{equation}\label{eq:7}
    \E_T \oosB - \E_R \oosA = O_p(\sqrt{P/T}) + o_p(P^{-1/2})
    + o_p(Q^{-1/2}).
  \end{equation}
\end{lem}
We can view $\E_T \oosB - \E_R \oosA$ as noise introduced by
approximating the performance of the full-sample estimates with that
of the partial-sample estimates. The key term on the RHS
of~\eqref{eq:7} in practice is $O_p(\sqrt{P/T})$---unless $P/T$ is
small, this noise dominates the OOS average, making it an inconsistent
estimator of $\E_T \oosB$. The $o_p(P^{-1/2})$ term is completely
unrestrictive in our applications because we will multiply the OOS
average by at most $\sqrt{P}$ (for the CLT). And the $o_p(Q^{-1/2})$
term has some implications for interpreting these results: we can only
measure the average performance of the forecasting models for an
extended period in the future, long enough that the future
observations are essentially independent of the current
information. For $Q=1$, for example, the dependence between
$y_{T+\h+1}$ and the current information set $\Fs_T$ may be very
strong.

Finally, we can use Lemmas~\ref{res-mixingale}
and~\ref{res-convergence} to show that the DMW test is asymptotically
normal and centered at $\E_T \oosB$, as long as
Assumption~\ref{asmp-4}b holds.

\begin{thm}\label{res-oost}
  Suppose that Assumptions \ref{asmp-1}--\ref{asmp-5} hold (including
  \ref{asmp-4}b), that $\gamma \to \infty$ and $\gamma/P \to 0$ as $T
  \to \infty$, and that $\sigma^2$ is uniformly a.s. positive. Define
  $\sh^2$ to be the usual OOS HAC estimator of the asymptotic variance
  of $\oosA$,
  \begin{equation}
    \sh^2 \equiv P^{-1} \oosSum{s,t}{1} (D_t - \oosA)(D_s - \oosA)
    \vWeight.
  \end{equation}
  Then
  \begin{equation}
    \sqrt{P}(\oosA - \E_T\oosB)/\sh \to^d N(0,1).
  \end{equation}
\end{thm}

The requirement that $\sigma^2$ be uniformly a.s. positive is not
restrictive under our asymptotic theory. Since the models'
coefficients are estimated with uncertainty in the limit, the two
models give different forecasts even if they both nest the DGP.  This
is intuitively similar to \citepos{GiW:06} rolling-window result, but
comes from different asymptotic theory; Giacomini and White keep the
variance of the OOS average positive by letting $P \to \infty$ with
$R$ fixed; in this paper, $R \to \infty$ but the variance remains
positive since $K \to \infty$ (quickly) as well.

If $K_1$ and $K_2$ were finite, West's (1996) and McCracken's (2007)
asymptotic theories would apply and the \oost\ test would remain normal as long
as $P/T \to 0$, even if $\sigma^2 \to 0$. Under that asymptotic
approximation, the variance will converge to zero whenever the true
DGP is nested in both of the forecasting models. The same principles
may also apply if $K_1$ and $K_2$ grow slowly, with $K_2/T \to
0$. It is likely that a second order expansion along the lines of
\citet{Mcc:07} would lead to asymptotic normality for that
intermediate case, but we leave that issue to future research.

Finally, Theorem~\ref{res:oostest} establishes that the DMW test can be
use to construct confidence intervals for $\E_T \oosB$ and test
hypotheses about $\E_T \oosB$. In particular, forecasters will often
want to test the null hypothesis that $\E_T \oosB \leq 0$, meaning
that the benchmark model is expected to be more accurate than the
alternative model in the future.

\begin{thm}\label{res:oostest}
  Suppose that the conditions of Theorem
  \ref{res-oost} hold. Then each of the usual Gaussian
  confidence intervals,
  \begin{gather}
    [\oosA - z_{\alpha/2} \, \sh /
    \sqrt{P}, \oosA + z_{\alpha/2} \sh / \sqrt{P}],\label{interval-twosided} \\
    [\oosA - z_{\alpha} \, \sh / \sqrt{P}, +\infty), \label{interval-greater}
    \intertext{and}
    (-\infty, \oosA + z_{\alpha} \, \sh / \sqrt{P}],
  \end{gather}
  contains $\E_T\oosB$ with probability $\alpha$ in the limit, with
  $z_{\alpha}$ the $1-\alpha$ quantile of the standard normal
  distribution. If, in addition, $\lim \Pr[\E_T \oosB \leq 0]$ is
  positive then
  \begin{equation}\label{eq:23}
    \lim \Pr[P^{1/2}\oosA/\sh > z_{\alpha} \mid \E_T
    \oosB \leq 0] \leq \alpha.
  \end{equation}
\end{thm}
\noindent
The results for confidence intervals in Theorem~\ref{res:oostest}
follow immediately from Theorem~\ref{res-oost},
but~\eqref{eq:23} requires additional proof.

\subsection{Failure of in-sample statistics for forecast evaluation}
\label{sec:insample}

In this subsection we look at the behavior of some full-sample
statistics under the same asymptotic theory as before.  We show that
these statistics, which include common tests such as the Wald test as
well as model selection criteria such as the AIC, do not measure the
models' generalization error and consequently do not indicate which
model will be more accurate in the future. Some of these statistics
will tend to choose the larger model regardless of which model will be
more accurate in the future, whereas others tend to choose the smaller
model.  This is a different issue than whether or not full-sample
tests are valid for testing hypotheses about the pseudotrue
coefficients of the models; as \citet{Cal:11c} and \citet{Ana:12}
demonstrate, variations of the Wald test can be valid for those
hypotheses under increasing-$K$ asymptotics.\footnote{%
  \citet{Ana:12} shows that the Wald test is invalid in general and
  gives an adjustment that corrects the critical values; he also shows
  that the $F$-test is asymptotically valid under certain conditions
  on the distribution of the regressors.  \citet{Cal:11c} shows that
  the $F$-test is asymptotically invalid without Anatolyev's
  constraint, even under homoskedasticity, and provides a correction
  that gives valid tests.  Both papers only consider independent
  observations.} %
To simplify the presentation and the intuition we only derive results
for nested models and for MSE loss, but the conclusions hold much more
generally.\footnote{%
  In this section, we define
  $M_1 = [I_{K_1} \ 0_{K_1 \times (K_2 - K_1)}]$ and
  $M_2 = [0_{(K_2 - K_1) \times K_2} \ I_{K_2 - K_1}]$ as the
  selection matrices that return the first $K_1$ and the last
  $K_2 - K_1$ elements of $\theta_2$, and assume that the regressors
  are ordered so that the first $K_1$ elements of $x_{2,t}$ correspond
  to $x_{1,t}$.} %

The full-sample statistics we study in this paper share a common
property. They choose the alternative model when the distance between
a subset of their coefficient estimates and the origin exceeds a
threshold, and choose the benchmark otherwise. For the Wald test, for
example, this threshold is chosen so that the test has correct size
when those coefficients are zero in population. For small models that
can be consistently estimated, these coefficient estimates are close
to their true values and so this criterion can be a reasonable proxy
for the relative accuracy of the larger model.

But for overfit models, the estimates will typically be far from both
their pseudotrue values and also from zero. In that case, the Wald test and
the AIC will both tend to choose the larger model even when it is less
accurate than the smaller benchmark model. This phenomenon is driven
by the dimensionality of the alternative model, since there are more
potential values of the coefficient estimator that are far from zero
when it has many elements. The threshold for the Wald test and the AIC
is set by construction to be a bounded distance from the origin, and
every other statistic that shares this property has the same behavior
and can reject the benchmark model with high probability, even when
it is more accurate. This behavior is
formalized in Theorem~\ref{res:insample1}.

\begin{thm}\label{res:insample1}
  Suppose Assumptions~\ref{asmp-1}--\ref{asmp-4} hold and let
  $\Lambda$ be a model selection statistic that takes the values zero
  or one; $\Lambda = 0$ indicates that the benchmark model is chosen
  and $\Lambda = 1$ indicates the alternative.  Moreover, assume that
  $L(e) = e^2$ and that there exist a deterministic scalar $c$
  and a sequence of possibly random matrices $V_T$ such that
  \begin{equation}\label{eq:25}
    \limsup_{T \to \infty}
    \Pr[\Lambda \geq \1\{\bh{2T}'M_2' V_T M_2\bh{2T} > c \}]
    \to 0
  \end{equation}
  where the eigenvalues of $V_T$ are uniformly bounded in probability,
  $\plim \rank(V_T)/T > 0$, and neither $V_T$ nor $c$ depend on
  $\theta$ or the value of $\bh{2T}$.

  Then there exist DGPs satisfying these assumptions such that
  \begin{equation}\label{eq:11}
    \liminf_{T \to \infty} \E( \Lambda \mid \E_T \oosB \leq 0 ) \geq 1/2.
  \end{equation}
\end{thm}

As we state above,~\eqref{eq:11} means that, with high probability,
the benchmark model is rejected even when it is more accurate. In
general, $\delta$ can be made arbitrarily small by increasing the
region enclosed by $\1\{\bh{2T}'M_2' V_T M_2\bh{2T}\}$ and DGPs can be
chosen to put the limiting quantity in~\eqref{eq:11}
arbitrarily close to 1 while still satisfying
the assumptions of the Theorem.  The DGPs and
statistics used in Theorem~\ref{res:insample1} are simple and
common. They include correctly specified linear models with normal and
homoskedastic errors; and the statistics that meet the restrictions on
$\Lambda$ include the $F$-test and AIC.

For example, the $F$-statistic for the null hypothesis that the
benchmark model is correctly specified is known to have the form
\begin{equation*}
  F = \frac{\bh{2T}' M_2' ( M_2 (X_{2T}'X_{2T})^{-1} M_2' )^{-1} M_2
    \bh{2T}}{s^2 (K_2 - K_1)}
\end{equation*}
where $s^2$ is the usual estimator of the variance of the regression
error. Then let $c$ be any number greater than 1 and define
\begin{equation*}
  V_T = M_2' \big( M_2 \big(\tfrac{1}{s^2 (K_2 - K_1)} X_{2T}'X_{2T}\big)^{-1}
  M_2' \big)^{-1} M_2'.
\end{equation*}
If Assumption~\ref{asmp-2} holds and $s^2$ is consistent then $V_T$
has uniformly bounded eigenvalues and has rank $K_2 - K_1$, so it
satisfies~\eqref{eq:25}.
Robust variations of the Wald test can obviously satisfy~\eqref{eq:25}
for similar reasons, and the AIC for nested linear models is
equivalent to using the $F$-test with the critical value %
$(e^{2 (K_2 - K_1) / T} - 1) \cdot (T - K_2) / (K_2 - K_1)$, %
which converges to a finite limit, so the AIC satisfies~\eqref{eq:25}
as well.

For statistics that don't satisfy~\eqref{eq:25}, the behavior can be
quite different. The BIC, for example, can also be written in terms of
the $F$-statistic and is equivalent to using the critical value $(T \,
e^{2 (K_2 - K_1) / T} - 1) \cdot (T - K_2) / (K_2 - K_1)$. This
critical value diverges as $T \to \infty$, ensuring that~\eqref{eq:25}
fails for any $c$. For this statistic, we have the opposite problem as
before: when the alternative model is \emph{more} accurate, the
coefficient estimates of the larger model are contained in a bounded
region of the parameter space. Since the acceptance region of the BIC
grows, it eventually contains \emph{any} bounded region of the
parameter space. For large enough $T$, the BIC will always choose the
\emph{smaller model}, even when the larger model is more accurate.

This behavior is formalized in Theorem~\ref{res:insample2}.
\begin{thm}\label{res:insample2}
  Suppose Assumptions~\ref{asmp-1}--\ref{asmp-4} hold, let $L(e) =
  e^2$, and let $\Lambda$ be a model selection statistic as in
  Theorem~\ref{res:insample1}. Also assume that, for any finite scalar
  $c$,
  \begin{equation}\label{eq:27}
    \Pr[\Lambda \leq \1\{\bh{2T}'M_2'J M_2\bh{2T} > c \}] \to 1
    \quad \text{as } T \to \infty.
  \end{equation}

  Then there exist DGPs satisfying these assumptions such that
  \begin{equation}\label{eq:15}
    \E( \Lambda \mid \E_T \oosB \geq 0 ) \to^p 0.
  \end{equation}
\end{thm}

The condition~\eqref{eq:27} requires that the acceptance region of
$\Lambda$ eventually contains any finite cylinder centered at the
origin. Again,~\eqref{eq:15} implies that statistics like the BIC
will always choose the smaller model for some DGPs, even when the
larger model will give more accurate forecasts. Both models may be
overfit, in that both $K_1/T$ and $K_2/T$ may both be positive in the
limit; the key is that $(K_2 - K_1)/T$ is also positive in the limit.

It is important to remember that previous research, such as
\citet{Cal:11c} and \citet{Ana:12}, does not predict these
results. When $\Lambda$ represents a test statistic, the test may have
correct size for the null hypothesis that the additional coefficients
on the larger model are zero. The results in this subsection are being
driven by the full-sample statistics' behavior when the smaller model
is misspecified but more accurate.

Any statistic that uses the distance of the models' estimated
coefficients from a set point (the origin being the most common) is
poorly suited for choosing between overfit forecasting models. These
models only forecast well when their coefficient estimates are close
to their pseudotrue values, which can be far from the origin or any
other prespecified
point. Depending on the statistic, it can be biased towards choosing
the larger model or the smaller model. Formal in-sample \emph{tests}
will likely be biased towards the larger model, as we show for the
$F$-test and Wald test in Theorem~\ref{res:insample1}.

\subsection{An extended simple example}
\label{sec:example}

This subsection illustrates the previous theoretical results with a
concrete example. Let $L(e) = e^2$ and $h = 1$. Suppose that the benchmark is
nested in the alternative and $(x_{2t}, \e_{2,t+1}) \sim i.i.d.\
N(0,I)$, and let $y_{t+1} = x_{2t}'\theta_2 + \e_{2,t+1}$
be the DGP. Also assume that $K_2/T \to c_2$ and $K_1 / T \to c_1 <
c_2$.

The first part of this subsection shows how the assumptions in
Section~\ref{sec:asmp} are satisfied and the second part
demonstrates asymptotic normality of the DMW \oost\ test. The third
part explicitly shows that $\E_R \oosA$ converges to $\E_T \oosB$ only
when the test sample is small. And the last part demonstrates that the
$F$-test does not indicate which model will be more
accurate in the future, even in this simple example.

\subsubsection*{Fulfillment of Assumptions 1--5}

We will go through the assumptions one by one. The first is
a condition on the dependence and moments of the process. Since the
DGP is i.i.d. Normal, these conditions are satisfied trivially.

Assumption~\ref{asmp-2} deals with the design matrix. In this example,
we directly assume that $K_2$ and $K_1$ grow at the correct rates so
the the assumption is satisfied. The variance of $y_{t+1}$ given
$\Fs_t$ is simply the unconditional variance of $\e_{2,t+1}$, which is
1; and the eigenvalues of $\E x_{2t} x_{2t}$ all equal 1 as well; and
so they are all uniformly positive and finite as required.

Assumption~\ref{asmp-2} also requires the eigenvalues of $S^{-1}
X_{iS}'X_{iS}$ to be positive and finite, and for the largest
eigenvalue of $S^{-1} X_{iS}'X_{iS}$ to be bounded in $L_3$.  These
results follow from developments in random matrix theory.
\citet{Gem:80} establishes that the largest eigenvalue of
$X_{iS}'X_{iS}$ is of order $S + K_i$ and \citet{Joh:01} shows that it
converges in distribution to the Tracy-Widom law of order 1
\citep{TrW:96}, which has finite 3rd moments. \citet{Sil:85} and
\citet{BFP:98} prove similar results for the smallest eigenvalue.
Assumption~\ref{asmp-2} also requires the largest eigenvalue of
$(X_{iS}'X_{iS})^{-1}$ be bounded in $L_3$; this should follow from a
similar argument, but we are unaware of any papers that explicitly
derive moments for this eigenvalue.

For the conditional expectation of %
$\sum_{s,t=U}^{V-1} \e_{i,t+1} \e_{i,s+1} x_{i,s} x_{i,t}'$,
independence implies that
\begin{equation*}
  \E\Big(
  \sum_{s,t=U}^{V-1} \e_{i,s+1} \e_{i,t+1} x_{is}x_{it}'
  \mBig
  x_{i1},\dots,x_{i,U-1};
  \sum_{s=U}^{V-1} x_{is} x_{is}';
  x_{i,V},\dots,x_{i,T-1}
  \Big)
  = \sum_{s=U}^{V-1} x_{is} x_{is}'.
\end{equation*}
The largest eigenvalue of this last matrix is of order $K_i + V - U$,
as discussed, and it has at most $V - U$ nonzero eigenvalues,
ensuring that both~\eqref{eq:28} and~\eqref{eq:8} hold.

Assumption~\ref{asmp-3} restricts the loss function and the realized
OOS loss. In this example, we have
\begin{equation*}
  D_t^* =^d D_t = (e_{2,t+1} + x_{2t}'\theta_2 - x_{1t}'\bh{1t})^2
  - (e_{2,t+1} + x_{2t}'(\theta_2 - \bh{2t}))^2
\end{equation*}
which has bounded $\rho$th moments by construction. The loss function
is differentiable and $L'(e) = 2 e$, so \eqref{eq:29} holds as well.

Finally, Assumption~\ref{asmp-4} deals with the choice of test and
training sample and holds automatically. And Assumption~\ref{asmp-5}
requires the kernel of the HAC variance estimator to be continuous at
zero; it is straightforward to construct an estimator that satisfies
this condition. However, in this section we will use the estimator
\begin{equation*}
  \sh^2 = \frac{1}{P} \sum_{t=R+1}^{T-1} (D_t - \oosA)^2
\end{equation*}
which does not satisfy Assumption~\ref{asmp-5} but nevertheless is
consistent for the conditional variance of $\oosA$ because the
underlying observations are independent.

\subsubsection*{Asymptotic normality of the OOS average, Lemma 1}

This subsection walks through Lemma 1's CLT. Since the observations in
this example are i.i.d., we do not need to use mixingale theory to
derive the results, but the OOS process is still affected by the
estimation error in $\bh{1t}$ and $\bh{2t}$ and has a high degree of
dependence. This dependence makes the OOS process an MDS, and MDS
asymptotic theory replaces mixingale theory in this example.

By construction, we have
\begin{equation*}
  D_t =
  \begin{cases}
    2 \e_{t+1} (x_{2t}'\bh{2R} - x_{1t}'\bh{1R})
    + (x_{2t}'\theta_2 - x_{1t}'\bh{1R})^2
    - (x_{2t}'\theta_2 - x_{2t}'\bh{2R})^2
    & \text{if } t < T \\
    2 \e_{t+1} (x_{2t}'\bh{2T} - x_{1t}'\bh{1T})
    + (x_{2t}'\theta_2 - x_{1t}'\bh{1T})^2
    - (x_{2t}'\theta_2 - x_{2t}'\bh{2T})^2
    & \text{if } t > T
  \end{cases}
\end{equation*}
and so we can explicitly derive the components of
Lemma~\ref{res-mixingale}. First,
\begin{multline*}
  D_t - \E_R D_t
  = 2 \e_{t+1} (x_{2t}'\bh{2R} - x_{1t}'\bh{1R})
  + \Big\{(x_{2t}'\theta_2 - x_{1t}'\bh{1R})^2
          - \E_R (x_{2t}'\theta_2 - x_{1t}'\bh{1R})^2 \Big\} \\
  - \Big\{(x_{2t}'\theta_2 - x_{2t}'\bh{2R})^2
          - \E_R (x_{2t}'\theta_2 - x_{2t}'\bh{2R})^2 \Big\}
\end{multline*}
for $t < T$. Since the underlying observations are i.i.d., $\E_{t-1}
D_t = \E_R D_t$ a.s.\ and consequently $\{D_t - \E_R D_t, \Fs_{t};
t=R+1,\dots,T-1\}$ is an MDS.

Although transformations of $D_t - \E_R D_t$ will obviously not be
MDSes in general, it should be clear that transformations of $D_t$
will be MDSes after subtracting their conditional mean; i.e.
\begin{equation*}
  \{g(D_t) - \E_R g(D_t); \Fs_t; t = R+1,\dots,T-1\}
\end{equation*}
is an MDS as long as
$g(D_t)$ has finite mean, but $g(D_t - \E_R D_t)$ is not. This MDS
result holds because $x_t$ and $y_{t+1}$ are independent of $\bh{1R}$
and $\bh{2R}$, so
\begin{equation*}\begin{split}
  \E_R g(D_t)
  &= \int g\Big((x_1'\bh{1R})^2 - (x_2'\bh{2R})^2
       + 2 y (x'\bh{2R} - x'\bh{1R})\Big) \, f(y, x) \, dx \, dy \\
  &= \E_{t-1} g(D_t)
\end{split}\end{equation*}
a.s., where $f$ is the density of $(y_{t+1}, x_{2t})$ and $x_1$
denotes the first $K_1$ elements of $x$. This result parallels our
results for mixingales in the general case.

As a result, $\oosA$ and $(1/P) \sum_{t=R+1}^{T-1} D_t^2$ both obey
LLNs: $\oosA \to^p \E_R \oosA$ and
\begin{equation*}
  (1/P) \sum_{t=R+1}^{T-1} D_t^2 \to^p \E_R D_T^2.
\end{equation*}
Moreover, these convergence results imply that $\sh^2 - \var_R
\sqrt{P} \oosA \to^p 0$. And so
\begin{equation*}
  \sqrt{P} \oosA/\sh \to N(0,1)
\end{equation*}
by the MDS CLT.

\subsubsection*{Convergence of $\E_R \oosA$ to $\E_T \oosB$, Lemma 2}

This subsection works through Lemma~\ref{res-convergence}. In this
example, the difference between $\E_R \oosA$ and $\E_T \oosB$ equals
\begin{equation*}
  \begin{split}
    \E_R \oosA - \E_T \oosB &= \Big\{(\bh{1R} - \theta_1)'(\bh{1R} -
    \theta_1) -
    (\bh{2R} - \theta_2)'(\bh{2R} - \theta_2)\Big\} \\
    & \quad - \Big\{(\bh{1T} - \theta_1)'(\bh{1T} - \theta_1) -
    (\bh{2T} - \theta_2)'(\bh{2T} - \theta_2)\Big\} \\
    &\begin{split}
      =\ep{T}'\Big\{
      & \tilde{X}_{1R}(X_{1R}'X_{1R})^{-2} \tilde{X}_{1R}'
      - X_{1T}(X_{1T}'X_{1T})^{-2} X_{1T}' \\
      & - \tilde{X}_{2R}(X_{2R}'X_{2R})^{-2} \tilde{X}_{2R}'
      + X_{2T}(X_{2T}'X_{2T})^{-2} X_{2T}'
      \Big\}\ep{T}.
    \end{split}
  \end{split}
\end{equation*}
with $\tilde{X}_{iR}$ the $T \times K_i$ matrix $[X_{iR}'\ 0]'$. This
last term is a quadratic form and the regressors and errors are
assumed to be normal, so we can find the rate that the difference
converges to zero in probability by calculating its first two moments.

The mean difference is
\begin{equation*}
  \begin{split}
  \E(\E_R \oosA - \E_T \oosB)
      &=O\Big(\max_i \E \tr \big\{
        \tilde{X}_{iR}(X_{iR}'X_{iR})^{-2} \tilde{X}_{iR}
        - X_{iT}(X_{iT}'X_{iT})^{-2} X_{iT} \big\} \Big) \\
      &=O\Big(\max_i \tr \big\{\E (X_{1R}'X_{1R})^{-1}
         - \E (X_{1T}'X_{1T})^{-1} \big\}\Big) \\
      &=O( K_1 P / (R-K_1)(T-K_1)) \\
      &=O(P/T)
    \end{split}
\end{equation*}
The first equality follows from the expectation of a quadratic form,
the second from routine manipulations of the trace operator, and the
third from the moments of the inverse Wishart distribution.

Similarly, the variance of the difference is
\begin{align*}
  \var(\E_R & \oosA - \E_T \oosB) \\ &=
  O\Big(\max_i \E\big[\tr((X_{iR}'X_{iR})^{-1} - (X_{iT}'X_{iT})^{-1})\big]^2
  + 2 \tr \E\big((X_{iR}'X_{iR})^{-2} - (X_{iT}'X_{iT})^{-2}\big) \\
  &\qquad - \big[\tr \E((X_{iR}'X_{iR})^{-1} - (X_{iT}'X_{iT})^{-1}) \big]^2 \Big)\\
  &= O(P/T)^2.
\end{align*}
So, in this example, $P^{1/2}(\E_R \oosA - \E_T \oosB) \to^p 0$
if $P^3/T^2 \to 0$, which is slightly weaker than the general
requirement that $P^2/T \to 0$.  If $\lim P^3/T^2 > 0$ the OOS average
still obeys the MDS CLT, but it is not centered correctly at $\E_T
\oosB$.

\subsubsection*{Behavior of the F-test}

This part illustrates the behavior of full-sample statistics in our
simple example. To simplify the presentation even more, assume that
the full-sample design matrix is orthogonal in-sample, not just in
population, so $X_{2T}'X_{2T} = T \cdot I_{K_2 \times K_2}$, and again
let $M_1$ and $M_2$ be the selection matrices for the first $K_1$ and
the last $K_2-K_1$ elements of $\theta_2$. In this example,
\begin{equation*}
  M_2 \bh{2T} \sim
  N(M_2 \theta_2, (1/T) \, I_{K_2-K_1})
\end{equation*}
and, conditional on $\E_T \oosB = 0$, the density of $M_2 \bh{2T}$
concentrates uniformly on the surface of the sphere centered at $M_2
\theta_2$ and passing through the origin:
\begin{equation*}
  (M_2 \bh{2T} - M_2 \theta_2)'(M_2 \bh{2T} - M_2 \theta_2) =
  \theta_2'M_2' M_2 \theta_2.
\end{equation*}
To see intuitively that it must pass through the origin, note that the
two models will give identical forecasts when $\bh{2T} = 0$ since the
regressors are orthogonal. (Identical forecasts obviously have the
same MSE.)

This region is a cylinder in the original space, $\Re^{K_2}$.  We can
also represent the null $\E_T \oosB \leq 0$ by conditioning on $\E_T
\oosB = -d$ for a fixed positive constant $c$. In that case, the
density of $M_2 \bh{2T}$ concentrates on the sphere
\begin{equation*}
  (M_2 \bh{2T} - M_2 \theta_2)'(M_2 \bh{2T} - M_2 \theta_2) =
  \theta_2'M_2' M_2 \theta_2 + d
\end{equation*}
which has the same center but larger radius.

For the $F$-test we have, as before,
\begin{equation*}
  F = \tfrac{T}{s^2 (K_2 - K_1)} \bh{2T}' M_2' M_2 \bh{2T}.
\end{equation*}
We know that $\sqrt{T}(F - 1)$ is asymptotically normal since its
numerator is Chi-squared with $K_2-K_1$ degrees of freedom and obeys a
CLT.  The test accepts if
\begin{equation}\label{eq:26}
  \tfrac{T}{s^2 (K_2 - K_1)} \bh{2T}' M_2' M_2 \bh{2T}
  \leq 1 + \delta / \sqrt{T},
\end{equation}
where $\delta$ is chosen to determine the size of the test. In other
words, the test accepts if $M_2 \bh{2T}$ falls in the sphere centered
at the origin with radius $s ((K_2 - K_1) / T)^{1/2} +
O_p(1/\sqrt{T})$.

% TikZ commands to draw circles
\newcommand{\circlefigA}[4]{
  \begin{tikzpicture}
    \clip (-#3,-#3) rectangle (#4,#4);
    \fill[lightgray] (-#3,-#3) rectangle (#4,#4);
    \foreach \r in {1.089629,1.190782,1.310824,1.466089,1.709963,2.71521}
      \draw [black,very thin] (1,1) circle ({veclen(\r,\r)});
    % rejection region for F-test
    \filldraw[fill=white,draw=black] (0,0) circle (#1);
    \begin{scope}
      \fill[darkgray] (0,0) circle (#1);
      \clip (0,0) circle (#1);
    \foreach \r in {1.089629,1.190782,1.310824,1.466089,1.709963,2.71521}
        \draw [black,very thin] (1,1) circle ({veclen(\r,\r)});
    \end{scope}
    % circle of equal generalization error
    \filldraw[fill=white,draw=black,thick] (1,1) let \p1=(1,1) in circle({veclen(\x1,\y1)});
    \draw (1,1) let \p1=(1,1) in circle({veclen(\x1,\y1)});
    \fill [black] (1,1) circle (2pt) node[right] {$M_2 \theta_2$};
    \draw[thick] (0,0) circle (#1);
    \draw (1,1)--(0,0);
    % axes
    \draw[->] (0,0)--(#2,0) node[right,fill=lightgray] {$e_1 M_2 \bh{2T}$};
    \draw[->] (0,0)--(0,#2) node[above,fill=lightgray] {$e_2 M_2 \bh{2T}$};
  \end{tikzpicture}
}

% TikZ commands to draw circles
\newcommand{\circlefigB}[4]{
  \begin{tikzpicture}
    \clip (-#3,-#3) rectangle (#4,#4);
    \fill[white] (-#3,-#3) rectangle (#4,#4);
    \draw[thick] (0,0) circle (#1);
    \filldraw[fill=lightgray,thick] (0.75,0.75) let \p1=(0.75,0.75) in circle({veclen(\x1,\y1)});
    \draw[thick] (0,0) circle (#1);
    \fill [black] (.75,.75) circle (2pt) node[right,fill=white] {$M_2 \theta_2$};
      \foreach \r in {0.9177846,0.8403243,0.765711,0.6923306,0.6185878,0.5427191,0.4621276,0.3723508,0.2625747,0.0617173,0.03120708}
        \draw [black,very thin] (0.75,0.75) circle (\r);
    \begin{scope}
      \clip (0.75,0.75) let \p1=(0.75,0.75) in circle({veclen(\x1,\y1)});
      \fill [black] (.75,.75) circle (2pt) node[right,fill=lightgray] {$M_2 \theta_2$};
      \filldraw[fill=darkgray,draw=black] (0,0) circle (#1);
      \clip (0,0) circle (#1);
      \foreach \r in {0.9177846,0.8403243,0.765711,0.6923306,0.6185878,0.5427191,0.4621276,0.3723508,0.2625747,0.0617173,0.03120708}
        \draw [black,very thin] (0.75,0.75) circle (\r);
      \draw[thick] (0.75,0.75) let \p1=(0.75,0.75) in circle({veclen(\x1,\y1)});
    \end{scope}
    \fill [black] (.75,.75) circle (2pt);
    \draw[->] (0,0)--(#2,0) node[right] {$e_1 M_2 \bh{2T}$};
    \draw[->] (0,0)--(0,#2) node[above] {$e_2 M_2 \bh{2T}$};
  \end{tikzpicture}
}

%%% Local Variables:
%%% mode: latex
%%% TeX-master: "../paper"
%%% End:

\begin{figure}
  \centering
  \begin{tabular}{cc}
  \subfloat[]{\circlefigA{1}{2.5}{1.4}{4.5}\label{fig:circleA}} &
  \subfloat[]{\circlefigB{1}{2.5}{1.4}{4.5}\label{fig:circleB}}
  \end{tabular}
  \caption{Graphs indicating the rejection region and region of equal
    generalization error for the models discussed in Section
    \ref{sec:insample}.  The smaller model has no estimated parameters
    and the larger has two coefficients.  The shaded regions in
    Figures (a) and (b) show the rejection region and the acceptance
    region respectively of the full-sample test given $\E_T \oosB \leq
    0$. Here $e_1 = (1,0)$ and $e_2 = (0,1)$ are the first and second
    selection vectors, so the horizontal axes represent changes in the
    first unique element of $\bh{2T}$ and the vertical axes represent
    changes in the second unique element.}
\label{fig:rreject}
\end{figure}

This is perhaps best illustrated with a picture.
Figure~\ref{fig:rreject} plots these quantities when $K_2 - K_1 =
2$. The circle centered at the origin plots the threshold for the
$F$-test; when $M_2 \bh{iT}$ falls outside this circle, the $F$-test
rejects. The circle centered at $M_2 \theta_2$ plots the set of points
for which $\E_T \oosB = 0$. The shaded region in
Figure~\ref{fig:rreject}~(a) plots the rejection region given $\E_T
\oosB \leq 0$ and the shaded region in Figure~\ref{fig:rreject}~(b)
plots the acceptance region given $\E_T \oosB \leq 0$.

In the picture, the true coefficients $\theta_2$ satisfy
\begin{equation*}
\theta_2'M_2'M_2\theta_2 > \var(\e_t)^{1/2} ((K_2 - K_1) / T)^{1/2}
\end{equation*}
so the center of the conditional distribution of $\bh{2T}$ given $\E_T
\oosB \leq 0$ is outside the acceptance region of the $F$-test. When
this is the case, the conditional probability that $\bh{2T}$ falls in
the rejection region is less than 1/2, since $M_2 \bh{2T}$ is
uniformly distributed on the cylinder
\begin{equation*}
  (M_2 \bh{2T} - M_2 \theta_2)'(M_2 \bh{2T} - M_2 \theta_2) =
  \theta_2'M_2' M_2 \theta_2 + d
\end{equation*}
for some positive $d$.

The AIC behaves essentially the same way. For the BIC, the radius of
the cylinder centered at the origin increases with $T$ and eventually
encompasses the cylinder centered at $M_2 \theta_2$, so it never
selects the alternative once $T$ is large enough.

%%% Local Variables:
%%% mode: latex
%%% TeX-master: "paper"
%%% TeX-command-extra-options: "-shell-escape"
%%% End:
