\section*{Supplemental Appendix (not for publication)}
\markright{Supplemental Appendix}

\noindent%
This appendix contains additional results to accompany
\citet{Cal:15b}. The first section presents a simple example with
i.i.d. observations that illustrates the main paper's theoretical
results. Many of these results can be proven directly under
independence and these proofs continue to demonstrate the key elements
of the main proofs. The second section presents additional Monte Carlo
simulations.

\section{An extended simple example}
\label{sec:example}

This section illustrates the previous theoretical results with a
concrete example. Let $L(e) = e^2$ and $h = 1$. Suppose that the benchmark is
nested in the alternative and $(x_{2t}, \e_{2,t+1}) \sim i.i.d.\
N(0,I)$, and let $y_{t+1} = x_{2t}'\theta_2 + \e_{2,t+1}$
be the DGP. Also assume that $K_2/T \to c_2$ and $K_1 / T \to c_1 <
c_2$.

The first part of this section shows how the assumptions in
Section~\ref{sec:asmp} are satisfied and the second part
demonstrates asymptotic normality of the DMW \oost\ test. The third
part explicitly shows that $\E_R \oosA$ converges to $\E_T \oosB$ only
when the test sample is small. And the last part demonstrates that the
$F$-test does not indicate which model will be more
accurate in the future, even in this simple example.

\subsection{Fulfillment of Assumptions 1--5}

We will go through the assumptions one by one. The first is
a condition on the dependence and moments of the process. Since the
DGP is i.i.d. Normal, these conditions are satisfied trivially.

Assumption~\ref{asmp-2} deals with the design matrix. In this example,
we directly assume that $K_2$ and $K_1$ grow at the correct rates so
the the assumption is satisfied. The variance of $y_{t+1}$ given
$\Fs_t$ is simply the unconditional variance of $\e_{2,t+1}$, which is
1; and the eigenvalues of $\E x_{2t} x_{2t}$ all equal 1 as well; and
so they are all uniformly positive and finite as required.

Assumption~\ref{asmp-2} also requires the eigenvalues of $S^{-1}
X_{iS}'X_{iS}$ to be positive and finite, and for the largest
eigenvalue of $S^{-1} X_{iS}'X_{iS}$ to be bounded in $L_3$.  These
results follow from developments in random matrix theory.
\citet{Gem:80} establishes that the largest eigenvalue of
$X_{iS}'X_{iS}$ is of order $S + K_i$ and \citet{Joh:01} shows that it
converges in distribution to the Tracy-Widom law of order 1
\citep{TrW:96}, which has finite 3rd moments. \citet{Sil:85} and
\citet{BFP:98} prove similar results for the smallest eigenvalue.
Assumption~\ref{asmp-2} also requires the largest eigenvalue of
$(X_{iS}'X_{iS})^{-1}$ be bounded in $L_3$; this should follow from a
similar argument, but we are unaware of any papers that explicitly
derive moments for this eigenvalue.

For the conditional expectation of %
$\sum_{s,t=U}^{V-1} \e_{i,t+1} \e_{i,s+1} x_{i,s} x_{i,t}'$,
independence implies that
\begin{equation*}
  \E\Big(
  \sum_{s,t=U}^{V-1} \e_{i,s+1} \e_{i,t+1} x_{is}x_{it}'
  \mBig
  x_{i1},\dots,x_{i,U-1};
  \sum_{s=U}^{V-1} x_{is} x_{is}';
  x_{i,V},\dots,x_{i,T-1}
  \Big)
  = \sum_{s=U}^{V-1} x_{is} x_{is}'.
\end{equation*}
The largest eigenvalue of this last matrix is of order $K_i + V - U$,
as discussed, and it has at most $V - U$ nonzero eigenvalues,
ensuring that both~\eqref{eq:28} and~\eqref{eq:8} hold.

Assumption~\ref{asmp-3} restricts the loss function and the realized
OOS loss. In this example, we have
\begin{equation*}
  D_t^* =^d D_t = (e_{2,t+1} + x_{2t}'\theta_2 - x_{1t}'\bh{1t})^2
  - (e_{2,t+1} + x_{2t}'(\theta_2 - \bh{2t}))^2
\end{equation*}
which has bounded $\rho$th moments by construction. The loss function
is differentiable and $L'(e) = 2 e$, so \eqref{eq:29} holds as well.

Finally, Assumption~\ref{asmp-4} deals with the choice of test and
training sample and holds automatically. And Assumption~\ref{asmp-5}
requires the kernel of the HAC variance estimator to be continuous at
zero; it is straightforward to construct an estimator that satisfies
this condition. However, in this section we will use the estimator
\begin{equation*}
  \sh^2 = \frac{1}{P} \sum_{t=R+1}^{T-1} (D_t - \oosA)^2
\end{equation*}
which does not satisfy Assumption~\ref{asmp-5} but nevertheless is
consistent for the conditional variance of $\oosA$ because the
underlying observations are independent.

\subsection{Asymptotic normality of the OOS average, Lemma 1}

This section walks through Lemma 1's CLT. Since the observations in
this example are i.i.d., we do not need to use mixingale theory to
derive the results, but the OOS process is still affected by the
estimation error in $\bh{1t}$ and $\bh{2t}$ and has a high degree of
dependence. This dependence makes the OOS process an MDS, and MDS
asymptotic theory replaces mixingale theory in this example.

By construction, we have
\begin{equation*}
  D_t =
  \begin{cases}
    2 \e_{t+1} (x_{2t}'\bh{2R} - x_{1t}'\bh{1R})
    + (x_{2t}'\theta_2 - x_{1t}'\bh{1R})^2
    - (x_{2t}'\theta_2 - x_{2t}'\bh{2R})^2
    & \text{if } t < T \\
    2 \e_{t+1} (x_{2t}'\bh{2T} - x_{1t}'\bh{1T})
    + (x_{2t}'\theta_2 - x_{1t}'\bh{1T})^2
    - (x_{2t}'\theta_2 - x_{2t}'\bh{2T})^2
    & \text{if } t > T
  \end{cases}
\end{equation*}
and so we can explicitly derive the components of
Lemma~\ref{res-mixingale}. First,
\begin{multline*}
  D_t - \E_R D_t
  = 2 \e_{t+1} (x_{2t}'\bh{2R} - x_{1t}'\bh{1R})
  + \Big\{(x_{2t}'\theta_2 - x_{1t}'\bh{1R})^2
          - \E_R (x_{2t}'\theta_2 - x_{1t}'\bh{1R})^2 \Big\} \\
  - \Big\{(x_{2t}'\theta_2 - x_{2t}'\bh{2R})^2
          - \E_R (x_{2t}'\theta_2 - x_{2t}'\bh{2R})^2 \Big\}
\end{multline*}
for $t < T$. Since the underlying observations are i.i.d., $\E_{t-1}
D_t = \E_R D_t$ a.s.\ and consequently $\{D_t - \E_R D_t, \Fs_{t};
t=R+1,\dots,T-1\}$ is an MDS.

Although transformations of $D_t - \E_R D_t$ will obviously not be
MDSes in general, it should be clear that transformations of $D_t$
will be MDSes after subtracting their conditional mean; i.e.
\begin{equation*}
  \{g(D_t) - \E_R g(D_t); \Fs_t; t = R+1,\dots,T-1\}
\end{equation*}
is an MDS as long as
$g(D_t)$ has finite mean, but $g(D_t - \E_R D_t)$ is not. This MDS
result holds because $x_t$ and $y_{t+1}$ are independent of $\bh{1R}$
and $\bh{2R}$, so
\begin{equation*}\begin{split}
  \E_R g(D_t)
  &= \int g\Big((x_1'\bh{1R})^2 - (x_2'\bh{2R})^2
       + 2 y (x'\bh{2R} - x'\bh{1R})\Big) \, f(y, x) \, dx \, dy \\
  &= \E_{t-1} g(D_t)
\end{split}\end{equation*}
a.s., where $f$ is the density of $(y_{t+1}, x_{2t})$ and $x_1$
denotes the first $K_1$ elements of $x$. This result parallels our
results for mixingales in the general case.

As a result, $\oosA$ and $(1/P) \sum_{t=R+1}^{T-1} D_t^2$ both obey
LLNs: $\oosA \to^p \E_R \oosA$ and
\begin{equation*}
  (1/P) \sum_{t=R+1}^{T-1} D_t^2 \to^p \E_R D_T^2.
\end{equation*}
Moreover, these convergence results imply that $\sh^2 - \var_R
\sqrt{P} \oosA \to^p 0$. And so
\begin{equation*}
  \sqrt{P} \oosA/\sh \to N(0,1)
\end{equation*}
by the MDS CLT.

\subsection{Convergence of $\E_R \oosA$ to $\E_T \oosB$, Lemma 2}

This section works through Lemma~\ref{res-convergence}. In this
example, the difference between $\E_R \oosA$ and $\E_T \oosB$ equals
\begin{equation*}
  \begin{split}
    \E_R \oosA - \E_T \oosB &= \Big\{(\bh{1R} - \theta_1)'(\bh{1R} -
    \theta_1) -
    (\bh{2R} - \theta_2)'(\bh{2R} - \theta_2)\Big\} \\
    & \quad - \Big\{(\bh{1T} - \theta_1)'(\bh{1T} - \theta_1) -
    (\bh{2T} - \theta_2)'(\bh{2T} - \theta_2)\Big\} \\
    &\begin{split}
      =\ep{T}'\Big\{
      & \tilde{X}_{1R}(X_{1R}'X_{1R})^{-2} \tilde{X}_{1R}'
      - X_{1T}(X_{1T}'X_{1T})^{-2} X_{1T}' \\
      & - \tilde{X}_{2R}(X_{2R}'X_{2R})^{-2} \tilde{X}_{2R}'
      + X_{2T}(X_{2T}'X_{2T})^{-2} X_{2T}'
      \Big\}\ep{T}.
    \end{split}
  \end{split}
\end{equation*}
with $\tilde{X}_{iR}$ the $T \times K_i$ matrix $[X_{iR}'\ 0]'$. This
last term is a quadratic form and the regressors and errors are
assumed to be normal, so we can find the rate that the difference
converges to zero in probability by calculating its first two moments.

The mean difference is
\begin{equation*}
  \begin{split}
  \E(\E_R \oosA - \E_T \oosB)
      &=O\Big(\max_i \E \tr \big\{
        \tilde{X}_{iR}(X_{iR}'X_{iR})^{-2} \tilde{X}_{iR}
        - X_{iT}(X_{iT}'X_{iT})^{-2} X_{iT} \big\} \Big) \\
      &=O\Big(\max_i \tr \big\{\E (X_{1R}'X_{1R})^{-1}
         - \E (X_{1T}'X_{1T})^{-1} \big\}\Big) \\
      &=O( K_1 P / (R-K_1)(T-K_1)) \\
      &=O(P/T)
    \end{split}
\end{equation*}
The first equality follows from the expectation of a quadratic form,
the second from routine manipulations of the trace operator, and the
third from the moments of the inverse Wishart distribution.

Similarly, the variance of the difference is
\begin{align*}
  \var(\E_R & \oosA - \E_T \oosB) \\ &=
  O\Big(\max_i \E\big[\tr((X_{iR}'X_{iR})^{-1} - (X_{iT}'X_{iT})^{-1})\big]^2
  + 2 \tr \E\big((X_{iR}'X_{iR})^{-2} - (X_{iT}'X_{iT})^{-2}\big) \\
  &\qquad - \big[\tr \E((X_{iR}'X_{iR})^{-1} - (X_{iT}'X_{iT})^{-1}) \big]^2 \Big)\\
  &= O(P/T)^2.
\end{align*}
So, in this example, $P^{1/2}(\E_R \oosA - \E_T \oosB) \to^p 0$
if $P^3/T^2 \to 0$, which is slightly weaker than the general
requirement that $P^2/T \to 0$.  If $\lim P^3/T^2 > 0$ the OOS average
still obeys the MDS CLT, but it is not centered correctly at $\E_T
\oosB$.

\subsection{Behavior of the F-test}

This part illustrates the behavior of full-sample statistics in our
simple example. To simplify the presentation even more, assume that
the full-sample design matrix is orthogonal in-sample, not just in
population, so $X_{2T}'X_{2T} = T \cdot I_{K_2 \times K_2}$, and again
let $M_1$ and $M_2$ be the selection matrices for the first $K_1$ and
the last $K_2-K_1$ elements of $\theta_2$. In this example,
\begin{equation*}
  M_2 \bh{2T} \sim
  N(M_2 \theta_2, (1/T) \, I_{K_2-K_1})
\end{equation*}
and, conditional on $\E_T \oosB = 0$, the density of $M_2 \bh{2T}$
concentrates uniformly on the surface of the sphere centered at $M_2
\theta_2$ and passing through the origin:
\begin{equation*}
  (M_2 \bh{2T} - M_2 \theta_2)'(M_2 \bh{2T} - M_2 \theta_2) =
  \theta_2'M_2' M_2 \theta_2.
\end{equation*}
To see intuitively that it must pass through the origin, note that the
two models will give identical forecasts when $\bh{2T} = 0$ since the
regressors are orthogonal. (Identical forecasts obviously have the
same MSE.)

This region is a cylinder in the original space, $\Re^{K_2}$.  We can
also represent the null $\E_T \oosB \leq 0$ by conditioning on $\E_T
\oosB = -d$ for a fixed positive constant $c$. In that case, the
density of $M_2 \bh{2T}$ concentrates on the sphere
\begin{equation*}
  (M_2 \bh{2T} - M_2 \theta_2)'(M_2 \bh{2T} - M_2 \theta_2) =
  \theta_2'M_2' M_2 \theta_2 + d
\end{equation*}
which has the same center but larger radius.

For the $F$-test we have, as before,
\begin{equation*}
  F = \tfrac{T}{s^2 (K_2 - K_1)} \bh{2T}' M_2' M_2 \bh{2T}.
\end{equation*}
We know that $\sqrt{T}(F - 1)$ is asymptotically normal since its
numerator is Chi-squared with $K_2-K_1$ degrees of freedom and obeys a
CLT.  The test accepts if
\begin{equation}\label{eq:26}
  \tfrac{T}{s^2 (K_2 - K_1)} \bh{2T}' M_2' M_2 \bh{2T}
  \leq 1 + \delta / \sqrt{T},
\end{equation}
where $\delta$ is chosen to determine the size of the test. In other
words, the test accepts if $M_2 \bh{2T}$ falls in the sphere centered
at the origin with radius $s ((K_2 - K_1) / T)^{1/2} +
O_p(1/\sqrt{T})$.

% TikZ commands to draw circles
\newcommand{\circlefigA}[4]{
  \begin{tikzpicture}
    \clip (-#3,-#3) rectangle (#4,#4);
    \fill[lightgray] (-#3,-#3) rectangle (#4,#4);
    \foreach \r in {1.089629,1.190782,1.310824,1.466089,1.709963,2.71521}
      \draw [black,very thin] (1,1) circle ({veclen(\r,\r)});
    % rejection region for F-test
    \filldraw[fill=white,draw=black] (0,0) circle (#1);
    \begin{scope}
      \fill[darkgray] (0,0) circle (#1);
      \clip (0,0) circle (#1);
    \foreach \r in {1.089629,1.190782,1.310824,1.466089,1.709963,2.71521}
        \draw [black,very thin] (1,1) circle ({veclen(\r,\r)});
    \end{scope}
    % circle of equal generalization error
    \filldraw[fill=white,draw=black,thick] (1,1) let \p1=(1,1) in circle({veclen(\x1,\y1)});
    \draw (1,1) let \p1=(1,1) in circle({veclen(\x1,\y1)});
    \fill [black] (1,1) circle (2pt) node[right] {$M_2 \theta_2$};
    \draw[thick] (0,0) circle (#1);
    \draw (1,1)--(0,0);
    % axes
    \draw[->] (0,0)--(#2,0) node[right,fill=lightgray] {$e_1 M_2 \bh{2T}$};
    \draw[->] (0,0)--(0,#2) node[above,fill=lightgray] {$e_2 M_2 \bh{2T}$};
  \end{tikzpicture}
}

% TikZ commands to draw circles
\newcommand{\circlefigB}[4]{
  \begin{tikzpicture}
    \clip (-#3,-#3) rectangle (#4,#4);
    \fill[white] (-#3,-#3) rectangle (#4,#4);
    \draw[thick] (0,0) circle (#1);
    \filldraw[fill=lightgray,thick] (0.75,0.75) let \p1=(0.75,0.75) in circle({veclen(\x1,\y1)});
    \draw[thick] (0,0) circle (#1);
    \fill [black] (.75,.75) circle (2pt) node[right,fill=white] {$M_2 \theta_2$};
      \foreach \r in {0.9177846,0.8403243,0.765711,0.6923306,0.6185878,0.5427191,0.4621276,0.3723508,0.2625747,0.0617173,0.03120708}
        \draw [black,very thin] (0.75,0.75) circle (\r);
    \begin{scope}
      \clip (0.75,0.75) let \p1=(0.75,0.75) in circle({veclen(\x1,\y1)});
      \fill [black] (.75,.75) circle (2pt) node[right,fill=lightgray] {$M_2 \theta_2$};
      \filldraw[fill=darkgray,draw=black] (0,0) circle (#1);
      \clip (0,0) circle (#1);
      \foreach \r in {0.9177846,0.8403243,0.765711,0.6923306,0.6185878,0.5427191,0.4621276,0.3723508,0.2625747,0.0617173,0.03120708}
        \draw [black,very thin] (0.75,0.75) circle (\r);
      \draw[thick] (0.75,0.75) let \p1=(0.75,0.75) in circle({veclen(\x1,\y1)});
    \end{scope}
    \fill [black] (.75,.75) circle (2pt);
    \draw[->] (0,0)--(#2,0) node[right] {$e_1 M_2 \bh{2T}$};
    \draw[->] (0,0)--(0,#2) node[above] {$e_2 M_2 \bh{2T}$};
  \end{tikzpicture}
}

%%% Local Variables:
%%% mode: latex
%%% TeX-master: "../paper"
%%% End:

\begin{figure}
  \centering
  \begin{tabular}{cc}
  \subfloat[]{\circlefigA{1}{2.5}{1.4}{4.5}\label{fig:circleA}} &
  \subfloat[]{\circlefigB{1}{2.5}{1.4}{4.5}\label{fig:circleB}}
  \end{tabular}
  \caption{Graphs indicating the rejection region and region of equal
    generalization error for the models discussed in Section
    \ref{sec:insample}.  The smaller model has no estimated parameters
    and the larger has two coefficients.  The shaded regions in
    Figures (a) and (b) show the rejection region and the acceptance
    region respectively of the full-sample test given $\E_T \oosB \leq
    0$. Here $e_1 = (1,0)$ and $e_2 = (0,1)$ are the first and second
    selection vectors, so the horizontal axes represent changes in the
    first unique element of $\bh{2T}$ and the vertical axes represent
    changes in the second unique element.}
\label{fig:rreject}
\end{figure}

This is perhaps best illustrated with a picture.
Figure~\ref{fig:rreject} plots these quantities when $K_2 - K_1 =
2$. The circle centered at the origin plots the threshold for the
$F$-test; when $M_2 \bh{iT}$ falls outside this circle, the $F$-test
rejects. The circle centered at $M_2 \theta_2$ plots the set of points
for which $\E_T \oosB = 0$. The shaded region in
Figure~\ref{fig:rreject}~(a) plots the rejection region given $\E_T
\oosB \leq 0$ and the shaded region in Figure~\ref{fig:rreject}~(b)
plots the acceptance region given $\E_T \oosB \leq 0$.

In the picture, the true coefficients $\theta_2$ satisfy
\begin{equation*}
\theta_2'M_2'M_2\theta_2 > \var(\e_t)^{1/2} ((K_2 - K_1) / T)^{1/2}
\end{equation*}
so the center of the conditional distribution of $\bh{2T}$ given $\E_T
\oosB \leq 0$ is outside the acceptance region of the $F$-test. When
this is the case, the conditional probability that $\bh{2T}$ falls in
the rejection region is less than 1/2, since $M_2 \bh{2T}$ is
uniformly distributed on the cylinder
\begin{equation*}
  (M_2 \bh{2T} - M_2 \theta_2)'(M_2 \bh{2T} - M_2 \theta_2) =
  \theta_2'M_2' M_2 \theta_2 + d
\end{equation*}
for some positive $d$.

The AIC behaves essentially the same way. For the BIC, the radius of
the cylinder centered at the origin increases with $T$ and eventually
encompasses the cylinder centered at $M_2 \theta_2$, so it never
selects the alternative once $T$ is large enough.

%%% Local Variables:
%%% mode: latex
%%% TeX-master: "paper"
%%% End:
