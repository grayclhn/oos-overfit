\documentclass[11pt]{article} \def\baselinestretch{1.08}
\usepackage{amssymb,amsmath,amsthm,harvard,graphicx, color, enumerate}
\usepackage[T1]{fontenc} \setcounter{MaxMatrixCols}{10}

% -- Margins for Editing --
\setlength{\voffset}{0.22in} \setlength{\hoffset}{-0.06in}
\setlength{\topmargin}{0in} \setlength{\headheight}{0in}
\setlength{\headsep}{0in} \setlength{\textheight}{7.33in}
\setlength{\oddsidemargin}{0in} \setlength{\textwidth}{5.67in}

\newcommand{\pp}[1]{#1} \newcommand{\intro}[1]{#1}
\newcommand{\context}{} \newcommand{\problem}{} \newcommand{\sowhat}{}
\newcommand{\solution}{} \newcommand{\point}[1]{#1}

\renewcommand{\qedsymbol}{\ensuremath \blacksquare}

%\newcommand{\pp}[1]{\textbf{#1}}
%\newcommand{\intro}[1]{\textsc{#1}}
%\newcommand{\context}{\subsubsection*{Shared Context}}
%\newcommand{\problem}{\subsubsection*{Statement of Problem}}
%\newcommand{\sowhat}{\subsubsection*{So What?}} 
%\newcommand{\solution}{\subsubsection*{Statement of Solution}}
%\newcommand{\point}[1]{\subsubsection*{Key Point} {#1}}

\input{../bib/definitions} \newtheorem{thm}{Theorem}[section]
\newtheorem{lem}[thm]{Lemma}
\newtheorem{claim}[thm]{Claim}
\newtheorem{cor}[thm]{Corollary}
\newtheorem*{BL}{Berbee's Lemma \cite{merlevede_coupling_2002}}

\theoremstyle{definition}
\newtheorem{asmp}{Assumption}

\theoremstyle{remark}
\newtheorem{rem}{Remark}
\newtheorem{example}{Example}

\numberwithin{equation}{section}

\DeclareMathOperator{\eqdist}{{}\overset{d}{=}{}}
\DeclareMathOperator{\p}{P}

\DeclareMathOperator{\pe}{P^*}
\DeclareMathOperator{\peT}{P_T^*}
\DeclareMathOperator{\peR}{P_R^*}

\DeclareMathOperator{\I}{1}

\DeclareMathOperator{\me}{E^*}
\DeclareMathOperator{\meR}{E^*_R}
\newcommand{\meRop}[1]{\meR({#1} \mid \b{}{R})}
\DeclareMathOperator{\meT}{E^*_T}

\newcommand{\m}[1]{\E \ft{#1}{T+1}}
\newcommand{\mr}[1]{\meR(\f{#1}{T+1}  \mid \b{}{R})}
\newcommand{\mt}[1]{\meT(\ff{#1}{T+1} \mid \b{}{T})}
\newcommand{\ct}[1]{\E(\ff{#1}{T+1} \mid \F{T})}
\newcommand{\mm}[2]{M_{#1}({#2})}
\newcommand{\mmr}[1]{M_{#1}(\b{#1}{R})}
\newcommand{\mmt}[1]{M_{#1}(\b{#1}{T})}
\newcommand{\mmm}[1]{M_{#1}(\B{#1})}

\newcommand{\mT}[1]{\E \ftT{#1}{T+1}}
\newcommand{\mrT}[1]{\meR(\fT{#1}{T+1}  \mid \bT{}{R_T})}
\newcommand{\mtT}[1]{\meT(\ffT{#1}{T+1} \mid \bT{}{T})}
\newcommand{\ctT}[1]{\E(\ffT{#1}{T+1} \mid \FT{T})}
\newcommand{\mmT}[2]{M_{{#1}T}({#2})}
\newcommand{\mmrT}[1]{\mmT{#1}{\bT{#1}{R_T}}}
\newcommand{\mmtT}[1]{\mmT{#1}{\bT{#1}{T}}}
\newcommand{\mmmT}[1]{\mmT{#1}{\BT{#1}}}
\newcommand{\ST}[1]{\Sigma_T({#1})}
\newcommand{\SrT}{\ST{\bT{}{R_T}}}

\newcommand{\F}[1]{\mathcal F_{#1}}
\newcommand{\FT}[1]{\mathcal F_{T,{#1}}}
\newcommand{\G}{\mathcal G}

\newcommand{\x}[2]{x_{{#1}{#2}}}
\newcommand{\xelem}[3]{x_{{#1}{#2}}^{#3}}
\newcommand{\y}[1]{y_{#1}}
\newcommand{\e}[2]{e_{{#1}{#2}}}
\newcommand{\eelem}[3]{e_{{#1}{#2}}^{#3}}
\newcommand{\f}[2]{L_{{#1}{#2}}}
\newcommand{\ff}[2]{L^T_{{#1}{#2}}}
\newcommand{\ft}[2]{L^*_{{#1}{#2}}}
\newcommand{\X}[2]{\boldsymbol{X}_{{#1}{#2}}}
\newcommand{\Y}[1]{\boldsymbol{Y}_{#1}}
\newcommand{\B}[1]{\theta_{#1}^*}
\renewcommand{\b}[2]{\hat \theta_{{#1}{#2}}}
\newcommand{\ep}[2]{\varepsilon_{{#1}{#2}}}
\newcommand{\Ep}[2]{\boldsymbol{\varepsilon}_{{#1}{#2}}}
\newcommand{\el}[1]{\bar L_{#1}}

\newcommand{\xT}[2]{x_{{#1}T,{#2}}}
\newcommand{\xTelem}[3]{x_{{#1}T,{#2}}^{#3}}
\newcommand{\yT}[1]{y_{T,{#1}}}
\newcommand{\eT}[2]{e_{{#1}T,{#2}}}
\newcommand{\eTelem}[3]{e_{{#1}T,{#2}}^{#3}}
\newcommand{\fT}[2]{L_{{#1}T,{#2}}}
\newcommand{\ffT}[2]{L^T_{{#1}T,{#2}}}
\newcommand{\ftT}[2]{L^*_{{#1}T,{#2}}}
\newcommand{\XT}[2]{\boldsymbol{X}_{{#1}T,{#2}}}
\newcommand{\YT}[1]{\boldsymbol{Y}_{T,{#1}}}
\newcommand{\BT}[1]{\theta^*_{{#1}T}}
\newcommand{\bT}[2]{\hat \theta_{{#1}T,{#2}}}
\newcommand{\epT}[2]{\varepsilon_{{#1}T,{#2}}}
\newcommand{\EpT}[2]{\boldsymbol{\varepsilon}_{{#1}T,{#2}}}
\newcommand{\elT}[1]{\bar L_{{#1}T}}

%%% Local Variables: 
%%% mode: latex
%%% TeX-master: "calhoun_fixed_window"
%%% End: 
\begin{document}

\title{Limit Theory for Overfit Forecasts} \author{Gray Calhoun}
\maketitle

\begin{abstract}
This paper proposes a new limit theory for statistics that evaluate
estimated models out-of-sample.  This theory implies out-of-sample
averages are asymptotically normal even when the models are nested and
are valid even when the models are overfit.  We introduce overfit by
assuming that the forecasts are produced by 
linear models and allowing the number of predictors, $K$, to increase
with the number of observations, $T$.  For at least one forecast, the
ratio $K/T$ converges to a positive constant, so in-sample statistics,
like the F-test, fail but, as we show, out-of-sample averages do not.
\end{abstract}
%%
\section{Introduction} 
%%
\context 
\intro{Consider two sequences of prediction errors, each of
  length $P$, the result of forecasting the same variable with two
  different estimated models.  The $R$ observations used to estimate
  the models are called, collectively, the estimation window, and the
  $P$ observations used to produce the errors are called the test
  sample.  There are $T$ observations in all, and $R + P = T$.
  \pp{This paper introduces a new asymptotic theory for the sample
  moments of these prediction errors that assumes at least one model
  is overfit, that statistics calculated from that model's sample
  residuals are unreliable because its estimated coefficients match
  the data too well.}}

If one of these models nests the other, a researcher can determine
whether the additional regressors help predict the dependent variable
(in population) by using one of two representative methods.  He or she
can run a regression over the full dataset and use a robust F-test to
directly test whether the coefficients on the extra predictors equal
zero.  Or he or she could instead produce a series of forecasts with each
model, and then test whether the models' predictive mean squared errors
(PMSE) are equal.  The first test is simple and easy, and the second
test is not; \citeasnoun{clark_tests_2001} and
\citeasnoun{mccracken_asymptotics_2007} show that the test statistic 
is not asymptotically normal and that the critical values must be
calculated by monte-carlo or bootstrap.  But the out-of-sample test
is much more popular among forecasters because it is believed to be
more reliable: see \citeasnoun{meese_empirical_1983},
\citeasnoun{stock_forecasting_2003} and
\citeasnoun{lettau_consumption_2001} for prominent examples. 

The asymptotic theory used to derive these out-of-sample
statistics, however, does not imply that they are any more reliable
than their in-sample counterparts.  There are two main approaches to
approximate these statistics' distributions.  The first, begun by
\citeasnoun{diebold_comparing_1995} and
\citeasnoun{west_asymptotic_1996} and extended to nested models by
\citeasnoun{clark_tests_2001}, \citeasnoun{chao_out_2001},
\citeasnoun{corradi_consistent_2002}, and
\citeasnoun{mccracken_asymptotics_2007}, finds the limit distribution
as though all of values of the models' coefficients were known, and
then adjusts that distribution to account for estimation
error.\footnote{Diebold and Mariano (1995) assume that the
  coefficients are known.  West (1996) 
  introduces the adjustment.} This adjustment requires the estimated
coefficients to be consistent and asymptotically normal.  But if they
were, a forecaster could instead just use an F-test directly on the
coefficients.

The second method, proposed by \citeasnoun{giacomini_tests_2006}, is
becoming a popular alternative to
\possessivecite{west_asymptotic_1996} 
approach because it always
gives an asymptotically normal statistic.  Their method
requires that the forecaster use a fixed-length rolling estimation
window --- the model for each forecast is estimated over the
previous $R$ observations --- and they derive a statistic to test
whether the difference between the models' forecasts is 
predictable.\footnote{Although \citeasnoun{giacomini_tests_2006} claim that
  their results also apply to fixed-length fixed window schemes, their
  proofs do not apply to that situation.  For a rolling window of
  length $R$, the period-$s$ prediction error only depends on the
  observations of period $s-R-1$ through period $s$, and so, for any 
  finite $R$, this sequence inherits mixing from the underlying
  observations.  But for a fixed window of length $R$, the
  period-$s$ prediction error depends on all of the past
  observations, periods $1$ through $s$, and so does not inherit
  mixing  from the underlying observations. Giacomini and White's
  proofs  require the prediction errors to be mixing.}
This rolling window can be appropriate if the series exhibit
substantial instability, so \possessivecite{giacomini_tests_2006}
out-of-sample 
statistic should be reliable in settings where an in-sample statistic
is not.  But, by its nature, their statistic measures the forecasting
performance of the estimated models, and does not reflect the
predictive content of their variables in population.  So Giacomini and
White's method suggests a new criterion to use to choose between
forecasting methods, but does not indicate why a researcher who wants
to study the true relationship between several series should use an
out-of-sample statistic.

Although the foundation of West's approach --- asymptotic
normality of the coefficient estimates --- implies that in-sample
and out-of-sample statistics should behave similarly, in practice they
do not.  \citeasnoun{stock_forecasting_2003}, for example, show that
in-sample tests of Granger Causality are much more likely to find
predictive relationships when predicting output and inflation than
out-of-sample tests. As \citeasnoun{inoue_in-sample_2005} argue, this
fact can support either approach: either the in-sample tests do not
preserve size, or the out-of-sample tests have low power.  The monte
carlo evidence is mixed.\footnote{This issue has been studied both
  analytically and in simulations by
  \citename{inoue_in-sample_2005} (\citeyear*{inoue_in-sample_2005},
  \citeyear*{inoue_selection_2006}),
  \citeasnoun{mccracken_data_1998}, \citeasnoun{clark_can_2004}, 
  \citeasnoun{clark_power_2005}, and \citeasnoun{chen_note_2005}.}

\solution
\intro{This paper introduces a new approximation for out-of-sample
  sample averages
  that does not require the coefficient estimates to be consistent or
  asymptotically normal.  This approximation allows us to study the
  behavior of out-of-sample statistics when one of the models is
  overfit, and we find that these statistics can be reliable even when
  in-sample statistics are not.  We impose overfit by studying
  the limit distribution of a sequence of linear regression models in
  which the ratio $K/T$ remains positive; $K$ is the number of
  predictors.  \citeasnoun{huber_robust_1973} shows that the OLS
  estimators are not asymptotically normal under this limit theory and
  that they have positive variance in the limit, so robust F-tests
  should be invalid. \pp{This new approximation
    gives evidence that the out-of-sample average loss is a reliable
    statistic for comparing complex, overfit models to a simple
    benchmark and that ad hoc critical values, such as
    \possessivecite{clark_tests_2001}, are not necessary; using the
    standard normal critical values gives valid but conservative
    tests.  The approximation also leads to a new criterion to use to
    evaluate a model's expected forecasting performance.}}

\pp{This increasing-dimension asymptotic theory is discussed in detail
  in Sections \ref{sec_asmp} and \ref{sec_lit}.}  Section
  \ref{sec_asmp} presents the notation, assumptions, and models that
  the rest of the paper will use, and Section \ref{sec_lit} studies
  the behavior of \possessivecite{west_asymptotic_1996} approximation
  under this new limit theory.  West's asymptotic distribution no
  longer holds because the
  test statistics are not centered on the expected loss of the
  pseudo-true models. Section \ref{sec_lit} also presents the results
  of a brief  
simulation demonstrating that conventional fixed-$K$ asymptotic theory
can be inaccurate in finite samples when $K/T$ is as small $0.03$.

\pp{Section \ref{sec_mr} presents our new approximation.}  
The fixed-window average loss converges to a measure of the expected
performance of the estimated models and is asymptotically normal
in most applications. Section \ref{sec_nested} shows that this average
can be used to compare nested models when the larger model is overfit, and that
using standard normal critical values leads to an asymptotically valid
(if conservative) test statistic.  

Section \ref{sec_perf} discusses the relationship between the
models' average loss and their future forecasting performance and
 shows that the conventional wisdom articulated by
 \citeasnoun{hastie_elements_2003} among others, that one should split
 the dataset in  half or in thirds is wrong: to estimate a model's
 future performance (under this paper's limit theory) $P/T$ must
 converge to zero and to construct a confidence interval, $P^2/T$ must
 converge to zero.

\pp{Section \ref{sec_mc} presents additional simulations.  The first
  simulation examines the size and power of naive one-sided tests for
  nested models and confirms that the true size is less than the
  nominal size.  The second simulation studies the use of the
  fixed-window average loss as a decision criterion for choosing a
  forecast.}

\point{
In short, this paper studies out-of-sample averages to see if they are
more robust to overfit than in-sample statistics and finds that they
are.  It shows that using a one-sided t-test to compare nested models
is conservative but preserves size as long as the benchmark is simple,
consistent with simulations presented in
\citeasnoun{clark_approximately_2007} but in 
contrast to existing theory.  It also shows that the test sample must
be much smaller than is commonly used if one hopes to accurately
compare the performance of the full-sample models that will be used to
produce real forecasts.} 
%%
\section{Notation and Assumptions} \label{sec_asmp}  
%%
This paper does not define a specific data generating process but
instead assumes that there are two competing forecasting models and both 
are misspecified.
\pp{The underlying data are represented as a stationary and absolutely
  regular stochastic array:
  \begin{equation} \label{eq_define_array}
  \{(\yT{t}, \xT{1}{t}, \xT{2}{t});\;t = 1,\dotsc,T+1;\;T\;\text{an integer}\},
  \end{equation}
and $\FT{t}$ denotes the information available in period $t$:
\[
\FT{t} \equiv \sigma(\yT{1},\xT{1}{1},\xT{2}{1},\dotsc,
\yT{t},\xT{1}{t},\xT{2}{t}).
\]}
The first model uses $\xT{1}{t}$ as predictors for $\yT{t}$ and the second
uses $\xT{2}{t}$. Each vector $\xT{j}{s}$ has $K_{jT}$ elements;
$K_{jT}$ can change with $T$, but is always assumed to be less than
$R_T$ (the sequence of estimation window sizes, $\{R_T\}$,
will be discussed in detail later in the paper).  Moreover,
$T$ and $R_T$ are implicitly assumed to be large enough that all of
the operations in this paper are well defined.  To keep the
presentation relatively clean, the $T$ subscript will be removed
whenever possible.  Although we only present results for one-period
forecasts, these results can be generalized easily to multi-period
forecasts. 

Assumption \ref{asmp_array} states the moment and dependence conditions
that the array \eqref{eq_define_array} must satisfy.
\begin{asmp} \label{asmp_array}
  The random array \eqref{eq_define_array} 
  is stationary and absolutely regular with coefficients $\beta_\tau$ of
  size $- \rho/(\rho-2)$; $\rho$ is greater than two and discussed further
  in Assumption \ref{asmp_loss_bound}.  The variance of $\yT{t}$ is
  uniformly positive and finite, and all of the eigenvalues of the
  covariance matrices of $\xT{j}{T}$ are uniformly positive and finite
  as well. 
  \qed
\end{asmp}

Each forecast is produced by a linear model; \pp{model $j$'s forecast for 
  period $T+1$ is $\x{j}{T+1}'\b{j}{T}$, with $\b{j}{S}$ the OLS
  estimate using observations one through $S$.}
The models' pseudo-true coefficients are denoted
$\B{j}$ and defined by
\[
\BT{j} \equiv \argmin_\theta \E (\yT{T+1} - \xT{j}{T+1}'\theta)^2.
\]
Assumption \ref{asmp_pseudo_true} rules out the uninteresting cases
where the forecast error vanishes.
\begin{asmp} \label{asmp_pseudo_true}
  The Euclidean norm of the pseudo-true coefficients satisfies
  $|\BT{j}|_2 = O(1)$, and the population residuals,
  $\epT{j}{t} \equiv \yT{t} - \xT{j}{t}'\BT{j}$, have uniformly positive and finite
  variance.
\qed
\end{asmp}

\pp{Since this paper will apply limit theory to the average loss over the test
sample, the observed test sample loss satisfies moment restrictions.}
The observed loss for model $j$ in period $s$ is $\f{j}{s}$ which is 
defined by
\[
\f{j}{s} = L(\e{j}{t}) \equiv L(\y{s} - \x{j}{s}'\b{j}{R});
\]
$L$ is the loss function of interest.  The period-$s$ loss for the
forecast produced by model $j$'s pseudo-true coefficients is
$\ft{j}{s} \equiv L(\ep{j}{s})$,
and the period-$s$ loss for the forecast produced by the full-sample
estimates is
$\ff{j}{s} \equiv L(\y{s} - \x{j}{s}'\b{j}{T})$.

The vector $\el{}$ denotes the average loss of the estimated models
over the test sample: 
\[
\el{j} \equiv P^{-1} \sum_{s=R+1}^T \f{j}{s}.
\]
Assumption \ref{asmp_loss_bound} restricts the moments of $\f{}{s}$.
\begin{asmp}\label{asmp_loss_bound}
The loss function $L$ is convex and there is a constant $B_L$
such that $\|\fT{j}{s}\|_\rho \leq B_L$ for all $j$, $s$, and $T$.
Moreover, the function $\E L(\yT{T+1} -
\xT{j}{T+1}'\theta)$ is continuously differentiable in $\theta$.
\qed
\end{asmp}

Finally, $|\cdot|_v$ is the $l_v$-norm for vectors in $\R^p$ 
($p$ arbitrary) and $\|\cdot\|_v$ the $L_v$-norm for $L_v$-integrable 
random variables.  The functions $\lambda_i(\cdot)$ take a 
square matrix argument
and return its $i$th eigenvalue.  All limits are stated for 
$T\to\infty$ unless explicitly labeled otherwise.
%%
\section{Background} \label{sec_lit}
%%
\intro{A large sample does not guarantee the accuracy of an asymptotic 
result; other factors come into play, among them the complexity of the model
estimated.  \citetwo{clark_using_2006}{clark_approximately_2007} show
that parameter estimation error 
affects the analysis of nested models in practice, making
\possessivecite{west_asymptotic_1996} 
limit theory unreliable.  
\pp{This section shows that the same problems occur without nesting and
motivates the use of increasing-$K$ asymptotics as a method of studying
that estimation error formally.}}\footnote{For this section,
assume that there is only one model under consideration.}

\intro{\pp{It is natural to use sequences of models with $K/T$
    positive to study the failure of in-sample asymptotics}} because
\pp{applied researchers choosing between models routinely make adjustments
  that are proportional to $K/T$.}  Statisticians have proposed model 
selection criteria, such as Mallow's $C_p$
\cite{mallows_comments_1973}, the AIC, \cite{akaike_information_1973}
and cross validation, because a model's {\it apparent error}, its 
average loss over the dataset used to estimate its unknown
    coefficients, is a biased estimator of the expected loss one encounters after using that 
model to predict new observations. The bias is proportional to $K/T$,
and these criteria are estimated by correcting the apparent error by 
a term proportional to $K/T$ (see, for example,
    \citename{efron_biased_1986} \citeyear*{efron_biased_1986},
    \citeyear*{efron_estimation_2004}). 

Moreover, \pp{since the variance of each element of $\b{}{T}$ is of
  order $T^{-1}$, the variance of this vector does not vanish if $K/T$
  remains positive.}  The behavior of M-estimators in this setting has
  been studied by Huber (1973), Yohai and Marona (1979), and Portnoy
  (1985, 1986).  If the coefficients are estimated by OLS, they are
  consistent and asymptotically normal only if $K/T$ converges to
  zero; if they are general M-estimates they require a faster rate of
  convergence.  By keeping $K/T$ positive, one can keep the variance
  of the coefficient estimates positive.  Since the coefficients are
  not asymptotically normal, the F-test is not necessarily
  asymptotically chi-square, so in-sample tests are unavailable.

  \intro{\pp{Expansions of $\el{}$ around the pseudo-true coefficients,
      like \possessivecite{west_asymptotic_1996} and
      \possessivecite{clark_tests_2001}, also do not hold when $K/T$ remains positive.}}
\pp{These approximations require that the coefficient
  estimates be root-$R$ consistent.} For a short illustration,
suppose that the underlying series are stationary; that the same
loss function is used to estimate the coefficients and evaluate
the forecasts; that $P$ and $R$ both equal $T/2$; and that a
fixed-window is used for evaluation.\footnote{These assumptions
  are not necessary, but simplify equation \eqref{eq_west}.} In this
special case, West's theory gives the approximation
\[
  P^{1/2} (\el{} - \E \ft{}{T+1})
  = P^{-1/2} \sum_{s=R+1}^T (\ft{}{s} - \E \ft{}{s})
  + o_p(R^{1/2} |\b{}{R} - \B{}|_2).
\]

Under conventional (fixed-$K$) asymptotic theory, $R^{1/2} |\b{}{R} -
\B{}|_2$ is tight, and the last term of the approximation converges to
zero in probability.  Then $P^{1/2}(\el{} - \E \ft{}{T+1})$ is
asymptotically normal as long as $P^{-1/2}\sum_s (\ft{}{s} - \E
\ft{}{s})$ is. But when $K$ grows proportionally to $T$, $R^{1/2}
|\b{}{R} - \B{}|_2$ is not tight. And since $\b{}{R}$ is not
asymptotically normal, the distribution of the remainder term is not
known.  In general, the variance of the fixed-window average remains
positive asymptotically, so $\el{}$ does not converge to any
non-random value.  Sections \ref{sec_mr} and \ref{sec_nested} will
explore this convergence in more detail. 

\intro{In practice, $K/T$ will always be positive, so this discussion
indicates that West's approximation is unreliable unless this ratio is
close to zero, but it is not necessarily clear how small the ratio
must be to make West's results accurate.  \pp{Some brief simulations
  demonstrate that 
the size of his tests can be too high even when $K/T$ is as small as
$0.03$, and empirically relevant values of $K/T$ seem to be from
$0.01$ to about $0.2$.}} Economists are sometimes interested in
studying tightly parameterized models, but they are often interested
in larger structural models.  \possessivecite{meese_empirical_1983}
seminal study 
of exchange rate models, for example, includes some models for which
the $K/T$ is almost 0.3. More recently,
\citeasnoun{stock_forecasting_2003} 
consider a range of output and inflation forecasts with this ratio
between 0.01 and 0.08, and \citeasnoun{del_negro_fit_2007} study
Bayesian DSGE models for which it is roughly 0.15.

\begin{table}[!tb]
    \label{tab_mc1stat}
    \begin{center}
    \caption{Simulated Size.  Nominal size is 10\%}
    \begin{tabular}{rrr@{.}lr@{.}lr@{.}lr@{.}l}
      & P = & \multicolumn{2}{r}{40} & \multicolumn{2}{c}{80}
      & \multicolumn{2}{c}{120} & \multicolumn{2}{c}{160} \\ \hline
      R = 80 && 15&4 & 18&5 & 19&0 & 21&0 \\
      160 && 12&0 & 14&5 & 15&9 & 17&2 \\ {}
    \end{tabular}
    \end{center}
    {\small
      According to \possessivecite{west_asymptotic_1996} theory, the random variable
      \eqref{eq_mc1stat} is asymptotically standard normal.  Each cell
      lists the percentage of simulations for which that r.v. exceeds
      $1.282$, the $90\%$-quantile of the standard normal
      distribution.}
\end{table}

To see whether West's theory is accurate for these values of $K/T$, we
ran a brief monte-carlo experiment to study forecasting with a
bivariate VAR(4); $T$
ranges from $120$ to $320$, so $K/T$ is between $0.025$ and $0.075$.
\pp{The DGP is taken from \citeasnoun{clark_approximately_2007} and is designed to
  represent macroeconomic forecasting.} Each entry in Table 1
  is the simulated size of a test at the 10\%-level of
  the null hypothesis that the expected PMSE of the bivariate VAR(4) is
  less than or equal to its true value.  We ran 3000 simulations: for
  each simulation we drew 320 observations from the stationary process
  with independent innovations:
\begin{align*}
  y_t &= 2.237 + 0.261 y_{t-1} + \varepsilon_{1t} \\
  z_t &= (0.804, - 0.221, 0.226, - 0.205) \cdot (z_{t-1},\dotsc, z_{t-4})
  + \varepsilon_{2t} \\
  \varepsilon &\sim N\left(\boldsymbol{0}, \begin{pmatrix}
      10.505 & 1.036 \\ \cdot & 0.366
      \end{pmatrix}\right).
\end{align*}
For each $R$ and $P$, we constructed sequences of forecasts, $\hat
y_t$, $t = R+1,\dotsc, R+P$ from the bivariate VAR(4):
\begin{equation*}
  \hat y_t = \hat \alpha_0 + \sum_{j=1}^4 \hat \alpha_j y_{t-j} 
  + \sum_{j=1}^4 \hat \beta_j z_{t-j}
\end{equation*}
with the coefficients estimated recursively by OLS. 

Each cell in Table 1 is the percentage of simulations for
which the random variable
\begin{equation} \label{eq_mc1stat}
P^{1/2} (\bar L - 10.505) / \hat \sigma
\end{equation}
is greater than $1.282$, with 
\[
\bar L \equiv P^{-1} \sum_{s=R+1}^{R+P} (y_s -\hat y_s)^2
\] 
and 
\[
\hat \sigma^2 \equiv (P-1)^{-1} \sum_{s=R+1}^{R+P} [(y_s -
\hat y_s)^2 - \bar L]^2.
\]
\pp{Under \possessivecite{west_asymptotic_1996} limit theory, this random variable is
  approximately standard normal, so each entry should be close to
  $10\%$; but the true size is higher, ranging from $12\%$ to $21\%$.} 
Since the parameter values of this simulation were chosen by Clark and West
(2007) to represent quarterly macroeconomic data, the values $R = 80$
and $P = 160$ roughly correspond to the common practice of estimating
forecasting models with pre-$1970$ data and assessing its performance
recursively from $1970$ on:\footnote{As in, for example, Stock and
  Watson (2003)} the true size is roughly twice the nominal size when
  testing this one-sided hypothesis with
such a procedure.  In general, increasing $P$ while keeping $R$
constant distorts the size more, suggesting that the variance of the
out-of-sample average decreases, but that the average is not centered at
$\E \ft{T+1}{}$.
%%
\section{The New Approximation} \label{sec_mr}
%%
Instead of using an expansion, we re-center each random sequence
$\{\f{}{s}\}$ by subtracting a random vector $\mmr{}$ from each term.
The new sequence is a mixingale, and $\mmr{}$ can be interpreted as a
conditional expectation after a change-of-measure.
  
\begin{lem} \label{lem_mixingale}
  Suppose that Assumptions \ref{asmp_array} to \ref{asmp_loss_bound}
  hold and define 
  \[
  \mmT{i}{\theta} \equiv \E L(\yT{T+1} - \xT{i}{T+1}'\theta).
  \]
  Then, for any $T$, any positive $j$, and any $\tau$ between zero and $j$,
  \begin{equation}
    \label{eq_mixingale_def}
    \| \E(\fT{i}{R_T+j} \mid \FT{R_T+j-\tau}) - \mmrT{i} \|_2 \leq 
     2^{1 + 1/\rho} B_L \; \zeta_{\tau}
  \end{equation}
  with $\zeta_\tau = O(\tau^{-1/2-\delta})$ for some positive
  $\delta$.  So the array 
  \[
  \{\fT{}{R_T+j} - \mmrT{}, \FT{R_T+j}\}
  \]
  is an $L_2$-mixingale array of size $-1/2$. 
  \qed
\end{lem}

The proof relies on a coupling argument due to
\citeasnoun{merlevede_coupling_2002} that builds on Berbee's Lemma
\cite{berbee_random_1979}.  \possessivecite{merlevede_coupling_2002}
statement of the Lemma is repeated here verbatim for reference.
\begin{BL}
  Let $X$ and $Y$ be random variables defined on $(\Omega, \mathcal T,
  \mathbb P)$ with values in a Polish space S.  Let $\sigma(X)$ be a
  $\sigma$-field generated by $X$ and $U$ be a random variable
  uniformly distributed on $[0,1]$ independent of $(X,Y)$.  Then there
  exists a random variable $Y^*$ measurable with respect to $\sigma(X)
  \vee \sigma(Y) \vee \sigma(U)$, independent of $X$ and distributed
  as $Y$, and such that
  \[
  \mathbb P(Y \neq Y^*) = \beta(X,Y).
  \] \qed
\end{BL}
The coefficient $\beta$ is the coefficient of absolute regularity.

\citeasnoun{merlevede_coupling_2002} use this result to bound the
$L_p$-norm of the distance between $Y$ and $Y^*$. We prove Lemma
\ref{lem_mixingale} by defining $(y^*, x^*)$ to satisfy:
\begin{enumerate}
\item $(y^*, x^*) \eqdist (\yT{R_T+j}, \xT{i}{R_T+j})$
\item $(y^*, x^*)$ is independent of $\FT{R_T+j-\tau}$.
\item $\p[(y^*, x^*) \neq (\yT{R_T+j}, \xT{i}{R_T+j})] = \beta_\tau$.
\end{enumerate}
Since $\mmrT{i} = \E(L(y^* - x^* \cdot \b{i}{R_T}) \mid
\FT{R_T+j-\tau})$ almost surely, the left side of
\eqref{eq_mixingale_def} is bounded by
\[
  \| \f{i}{R_T+j} - L(y^* - x^* \cdot \b{i}{R_T}) \|_2.
\]
We can directly use Merlevede and Peligrad's method of
proof to bound this last distance.

An immediate consequence of Lemma \ref{lem_mixingale} is that $\elT{i} -
\mmrT{i}$ converges to zero almost surely as $P_T$ increases.  We can
also show that $\sqrt{P_T} [\elT{} - \mmrT{}]$ is asymptotically
normal under a slightly stronger condition that its asymptotic
variance is positive definite.
\begin{lem} \label{lem_mixingale_normality}
  Suppose that the conditions of Lemma \ref{lem_mixingale} hold and that
  \[
  \lambda_{min}(\SrT)^{-1} = O_p(1),
  \] 
  with each element of $\ST{\cdot}$ defined by
  \[\left[\ST{(\theta_1,\theta_2)}\right]_{ij} \equiv P^{-1} \sum_{s,t=R+1}^T 
  \left[ \E L(\yT{s}-\xT{i}{s}'\theta_i) L(\yT{t}-\xT{j}{t}'\theta_j)
  - \mmT{i}{\theta_i} \mmT{j}{\theta_j}\right].
  \]
  If $P_T \to \infty$ as $T \to \infty$, then
  \begin{equation}
    P_T^{1/2} \SrT^{-1/2} \big[\elT{}  - \mmrT{}\big] \xrightarrow{d}
    N(0,I) \quad \text{as $T \to \infty$}. 
\end{equation}
\qed
\end{lem}
The proof is based on \possessivecite{de_jong_central_1997} Theorem 1,
a mixingale Central Limit Theorem.

This mixingale approximation allows us to work with sequences of
estimators that do not converge.  As in
\citeasnoun{giacomini_tests_2006}, such sequences can ensure that the
asymptotic variance matrix remains positive definite, even for nested
models. 

It is useful to compare this approximation to
\possessivecite{mccracken_robust_2000}. McCracken assumes that the
estimators $\bT{}{R_T}$ are asymptotically normal and that the
function $\mmT{}{\cdot}$ is smooth enough that
\[
\sqrt{R_T} [\mmrT{i} - \mmmT{i}]
\]
is asymptotically normal as a consequence of the delta-method.  As a
result, he, like \citeasnoun{west_asymptotic_1996}, can apply a
central limit theorem to $P_T^{1/2} [\elT{} - \mmmT{}]$
and then adjust its covariance matrix to account for the difference
between $\mmrT{i}$ and $\mmmT{i}$.

Lemmas \ref{lem_mixingale} and \ref{lem_mixingale_normality} show that
we can apply a Central Limit Theorem directly to $P_T^{1/2} [\elT{} -
\mmrT{}]$.  This extra generality allows us to study how out-of-sample
averages perform when their models are estimated imprecisely.
However, we need to impose more restrictions before we can relate
$\mmrT{}$ to objects that a researcher should be interested in, such
as $\mmmT{}$.
%%
\section{Comparing Nested Models} \label{sec_nested}  
In most applications, the benchmark model is very simple (a random walk,
for example) so we can treat the smaller model as having fixed $K$.
Only the alternative model is complex.
In this case, we can construct a conservative one-sided test for the
null hypothesis that smaller model is more accurate (in terms of MSE)
in population. 

\begin{thm}\label{thm_nesting1}
  Suppose that the conditions of Lemmas \ref{lem_mixingale} and
  \ref{lem_mixingale_normality} hold and that
  \begin{enumerate}
  \item[(i)] $K_{1T} = K_1$ and does not change with $T$; $K_{2T}/T \to
    k_2 > 0$; and $P_T/R_T \to \pi < \infty$. 
  \item[(ii)] $\hat \sigma_T$ is a consistent estimator of $(1,-1) \Sigma_T
    (1,-1)'$.
  \item[(iii)] Each of the smaller model's predictors is also used by
  the larger model (model two).
  \item[(iv)] $L(e) = e^2$.
  \end{enumerate}
  Then, under the null hypothesis
  \[
  H_0: \qquad \mT{1} = \mT{2} \qquad \text{for all $T$,}
  \]
  the one-sided t-test satisfies
  \begin{equation*}
    \lim_{T\to\infty} \p\Big[P_T^{1/2} (\elT{1} - \elT{2}) / \hat \sigma_T  \geq
    z_\alpha \Big] \leq \alpha,
  \end{equation*}
  with $z_\alpha$ the $(1-\alpha)$-quantile of the standard normal
  distribution.
  \qed
\end{thm}
In short, one-sided tests for nested models that (erroneously) act as
though the out-of-sample average were normal with mean $\m{}$ are
asymptotically valid.

This theorem relies on the inequality
\begin{multline} \label{eq_nesting_inequality}
  (\el{1} - \el{2}) - (\m{1} - \m{2}) \leq \\
\big[\el{1} - \mmr{1}\big] - \big[\el{2} - \mmr{2}\big] + o_p(R^{-1/2}),
\end{multline}
which holds because $\m{1} = \mmr{1} + o_p(R^{-1/2})$ and $\m{2} \leq
\mmr{2}$.

We should discuss the variance estimator, $\hat \sigma_T$, further.
Since the asymptotic variance, $\SrT$, is a random element that
depends on $\bT{}{R_T}$, the usual proofs that HAC estimators are
consistent do not apply.  Moreover, those proofs require NED
sequences, not mixingales, so they would not apply anyway.  Because of
the special structure of our mixingale process, though, it is
straightforward to prove the consistency of HAC estimators using a
coupling argument similar to the one used to prove Lemmas
\ref{lem_mixingale} and \ref{lem_mixingale_normality}.  In fact, one
can simply mimic the available NED proofs.  In this paper, we will
simply assume
the existence of a consistent estimator.  Lemma
\ref{lem_cov_matrix_appendix} (in the Appendix) contains the basic
argument for how to modify existing proofs,
\citename{davidson_consistency_1998}
(\citeyear*{davidson_consistency_1998},
\citeyear*{davidson_consistency_2000}) in particular.

In this case, there are more basic assumptions that guarantee that the
covariance matrix $\SrT$ is uniformly positive definite.  Remember
that the null hypothesis imposes that the coefficients on the
additional predictors used by the larger model are zero.  Under
Theorem \ref{thm_nesting1}'s assumptions, the models' prediction errors
satisfy the relationship
\begin{align*}
  \e{1}{s} & = \ep{1}{s} + o_p(1) \\
  \e{2}{s} & = \ep{1}{s} + z_{s}'\hat \alpha_{R_T} + o_p(1)
\end{align*}
with $z_s$ the additional predictors and $\hat \alpha_{R_T}$ their
coefficient estimates.  As long as $z_s$ and $\hat \alpha_{R_T}$ are
almost surely not zero, the two forecasts are almost surely
different.

This example is generalized slightly by the next lemma.  
\begin{lem} \label{lem_pd_conditions}
  Suppose that the conditions of Lemma \ref{lem_mixingale} and the
  additional conditions of Theorem \ref{thm_nesting1} hold.
  In addition, suppose that
  \begin{itemize}
  \item[(i)]  The maximum eigenvalues of
    $R_T^{-1} \XT{i}{R_T}'\XT{i}{R_T}$ and
    $R_T^{-1} \XT{i}{R_T}'\Ep{i}{R_T}\Ep{i}{R_T}'\XT{i}{R_T}$ are $O_p(1)$ and
    their minimum eigenvalues are bounded away from zero in probability.
  \item[(ii)] The first model's innovations, $\epT{1}{t}$, are
    sequentially exogenous and independent of $\xT{1}{s}$ and
    $\xT{2}{s}$ for $s = 1,\dotsc, t$.
  \item[(iii)] The elements of $\xT{2}{t}$ and $\bT{2}{R_T}$ are
    continuous random variables.
  \end{itemize}
  Then $[(1,-1) \SrT (1,-1)']^{-1} = O_p(1)$.
  \qed
\end{lem}
These eigenvalue conditions are analogous to the usual restrictions
made on asymptotic variance matrices.  The strong assumption of
sequential exogeneity simplifies the proof but is not crucial.  The
fact that the random variables are continuous rules out the
possibility that, for example, the estimation error,
$\xT{2}{T+1}'(\bT{2}{R_T} - \BT{2})$, is zero.

To summarize this section: researchers can use fixed-window
out-of-sample averages to compare the pseudo-true performance of an
overfit model to a simple benchmark.  If the benchmark is also overfit
(i.e., $K_{1T}/T$ remains positive), this approach does not work
because inequality \eqref{eq_nesting_inequality} does not hold.  Also,
if different loss functions are used to estimate and evaluate the
model, this approach again does not work because inequality
\eqref{eq_nesting_inequality} does not hold.  But Theorem
\ref{thm_nesting1} and Lemma \ref{lem_pd_conditions} 
justify using the naive out-of-sample one-sided t-test for most
empirical research.
%%
\section{Comparing Finite-Sample Performance} \label{sec_perf} 
%%
This section gives conditions under which $\mmrT{}$ converges to
$\mmtT{}$ and suggests that $\mmtT{}$ be used as a criterion for
choosing between forecasting models.

\pp{Ideally, a forecaster choosing between two models to use for period
  $T+1$ would chose the one that minimizes the expected loss given the
  information available in period $T$, $\ct{j}$.}  Even if the
underlying random variables are independent, $\F{T}$ has valuable
information about the models' performance --- the values of the
coefficient estimates $\b{1}{T}$ and $\b{2}{T}$. But usually a model
will use lagged variables as predictors, and their values are also
included in $\F{T}$. Consequently, a
forecaster choosing the model that minimizes the conditional
expectation will make better forecasts than one who minimizes the
unconditional expected loss, $\E \ff{j}{T+1}$.

For i.i.d. series, $\mmtT{} = \ct{}$ almost surely.  With dependence,
$\mmtT{}$ ignores the past information beyond the value of the
coefficient estimate, and so is a biased proxy for the true
conditional expectation.  However, the true conditional expectation
can be difficult to estimate, and the fixed-window out-of-sample
average can be used to estimate $\mmtT{}$.

\begin{lem} \label{lem_muR}
  Suppose that Assumptions \ref{asmp_array} to \ref{asmp_loss_bound}
  hold and that 
  \begin{itemize}
  \item[(i)] For each sequence $\{s_T\}$ with $s_T$ between $R_T$ and $T$, the
  maximum eigenvalues of $s_T^{-1} \XT{i}{s_T}'\XT{i}{s_T}$ and
  $s_T^{-1} \XT{i}{s_T}'\Ep{i}{s_T}\Ep{i}{s_T}'\XT{i}{s_T}$ are
  $O_p(1)$ and their minimum eigenvalues are bounded away from zero in
  probability.
  \item[(ii)] $P_T/R_T \to 0$, $K_{1T}/T \to k_1$, and $K_{2T}/T \to
  k_2$.  Both $k_1$ and $k_2$ are less than one.
  \item[(iii)] $L$ has finite left- and right-derivatives at every
  point, denoted $D_L(\cdot)$ and $D_R(\cdot)$ respectively.
  \end{itemize}
  Then
  \[
  \mmrT{i} - \mmtT{i} = O_{L_1}(\sqrt{P_T/R_T}),
  \]
  and so $\elT{} - \mmtT{} \to 0$ in probability.
  \qed
\end{lem}

Theorem \ref{thm_muR} is an immediate corollary.
\begin{thm} \label{thm_muR}
  Suppose the conditions of Lemmas \ref{lem_mixingale_normality}
  and \ref{lem_muR} hold.  If $P_T^2/R_T \to 0$, then
  \[
  P_T^{1/2} \SrT^{-1/2} \left[\elT{} - \mmtT{}\right] \xrightarrow{d}
  N(0, I) \quad \text{as $T \to \infty$.}
  \]
  \qed
\end{thm}

These results suggest two new ideas.  When the
models are overfit, out-of-sample averages implicitly condition on the
coefficient estimates.\footnote{\citeasnoun{efron_biased_1986} makes a
similar point about cross-validation in finite samples.  Our result,
though, is the first that we are aware of to study such conditioning
asymptotically.}  In economic forecasting applications, such
conditioning is desirable --- we will make predictions for the same
series using those coefficients, so averaging the forecasts'
performance over other hypothetical values for those coefficients is
inappropriate.  However, unless $P/R$ is very small, the forecasts
comparative performance can change when the models are re-estimated
over the entire dataset.  Since, in practice, $P$ must also be large
enough to justify a Law of Large Numbers or Central Limit Theorem,
these out-of-sample statistics may have limited use in macroeconomics.
%%
\section{Simulations} \label{sec_mc}
In progress.
\section{Conclusion} 
By studying the behavior of the fixed-window out-of-sample average
under a new limit theory that increases the number of predictors with
the number of observations, this paper shows that these out-of-sample
tests can prevent overfit and are properly sized when in-sample tests
are not.  Many of the previously known  results on these statistics do not
carry over to this setting, though: the performance of the model's
pseudo-true coefficients can not be estimated, but researchers can
still construct some one-sided confidence intervals; nested
comparisons are asymptotically normal; and the test sample must be
extremely small if this out-of-sample exercise will estimate how
well the models perform when they are re-estimated over the
full dataset.

Future research should study whether it is possible to improve the
power of out-of-sample tests while preserving size under this
asymptotic theory in the manner of
\citetwo{clark_using_2006}{clark_approximately_2007}; whether
resembling techniques can improve the restrictions on $P/R$; and how
these results can be extended to M-estimators and nonlinear models.
%%
\appendix
\section{Additional Technical Results and Proofs}
%%
\newcommand{\insum}{P_T^{-1/2}\sum_{s = R_T + (i-1)b_T + 1}^{R_T + i b_T}}
\newcommand{\insump}{P_T^{-1/2}
  \sum_{s = R_T + \lfloor P_T / b_T \rfloor b_T + 1}^T}
\newcommand{\outsum}{\sum_{i=1}^{\lfloor P_T/b_T \rfloor}}
\newcommand{\outsump}{\sum_{i=2}^{\lfloor P_T/b_T \rfloor + 1}}

\renewcommand{\L}{\mathcal L}

\begin{lem} \label{lem_coupling}
 Suppose Assumptions \ref{asmp_array} to \ref{asmp_loss_bound} hold.
 Then, for any $T$, $s$, $t$, and $u$, with $t \geq s > u \geq R_T$
 there exists an array $\{\tilde L_{jv}; v = s, \dotsc, t;\, j =
 1,2\}$  such that
 \begin{equation} \label{eq_coupling1}
   \E\left(\phi(\tilde L_s, \dotsc, \tilde L_t) \mid \FT{u} \right) = 
   \int \phi(\fT{}{s}, \dotsc, \fT{}{t}) \p(d\xT{}{s}, d\yT{s}, \dotsc,
   d\xT{}{t}, d\yT{t})
 \end{equation}
 almost surely for all measurable functions $\phi$ such that the expectations
 are finite.  Moreover, 
\begin{equation} \label{eq_coupling2}
  \p[\tilde L_{jv} \neq \fT{j}{v} \text{ for at least one $v$ and $j$}] =
 \beta_{s-u}
\end{equation}
 and
\begin{equation} \label{eq_coupling3}
  \| \tilde L_v - \fT{}{v} \|_2 \leq 2^{1+1/\rho} B_L
  \beta_{s-u}^{(\rho-2)/2\rho} \quad \text{for each $v$ and $j$.}
\end{equation}
\end{lem}

\begin{proof}
  \newcommand{\A}{\mathcal A}
  Fix $T$, $t$, $s$, and $u$.  The array $\A \equiv
  \{(\yT{\tau},\xT{}{\tau}, \dotsc, \yT{\tau + t-s}, \xT{}{\tau +
  t-s}); \; \tau\}$ is also absolutely regular of size
  $\rho/(\rho-2)$, so Berbee's Lemma allows
  us to construct a new array $\A^* \equiv \{( y_\tau^*,x_\tau^*,
  \dotsc, y_{\tau + t-s}^*, 
  x_{\tau + t-s}^*)\}$ that is independent of $\FT{u}$, equal to
  $\A$ in distribution, and satisfies $\p[\A^* \neq \A] =
  \beta_{s-u}$.

  Now it is easy to construct $\{\tilde L_{jv}\}$:
  \begin{equation}
    \tilde L_{jv} \equiv L(y_v^* - x_{jv}^* \cdot \bT{j}{R_T}), \quad j
    = 1,2, \quad v = s, \dotsc, t.
  \end{equation}
  Equations \eqref{eq_coupling1} and \eqref{eq_coupling2} are
  satisfied by construction, so it remains to prove
  \eqref{eq_coupling3}.  But \eqref{eq_coupling3} follows immediately
  from \possessivecite{merlevede_coupling_2002} Proposition 2.3 ---
  this proposition only uses (\ref{eq_coupling2}) and moment
  restrictions, not the equality of distributions.  As noted by
  \citeasnoun{dedecker_new_2005}, \citename{merlevede_coupling_2002}'s
  constant, $2^{p+2}$ can be reduced when $p=2$.
\end{proof}

\begin{lem} \label{lem_cov_matrix_appendix}
  Suppose $\{b_T\}$ is a sequence of positive integers such that $b_T
  \to \infty$ and $b_T/P_T \to 0$, and define 
  \[
  Z_{Ti} \equiv \insum [\fT{}{s} - \mmrT{}].
  \]
  If Assumptions \ref{asmp_array} to \ref{asmp_loss_bound} hold
  then
  \[
  \outsum \big[Z_{Ti}^2 - \meR(Z_{Ti}^2 \mid \FT{R_T})\big]
  \to 0 \quad \text{in $L_1$,}
  \]
  where $\meR$ is the integral with respect to a new probability
  measure $\peR$ that imposes independence between $\FT{R_T}$ and the
  sigma-field generated by the random variables 
  \[
  \{\yT{R_T+j}, \xT{1}{R_T+j}, \xT{2}{R_T+j};\; j>0\}
  \]
  but otherwise preserves the original probability measure.
\end{lem}

\begin{proof}
Much of the proof mimics that of \possessivecite{de_jong_central_1997}
Lemma 5.  For clarity, suppose that $Z_{Ti}$ is a scalar.  Define the
function
\[
h_c(x) = \begin{cases}
    \sgn(x) c \sqrt{b_T/P_T} & \text{if } |x| > c \sqrt{b_T/P_T} \\
    x & \text{otherwise}
\end{cases}
\]
for an arbitrary constant $c$.  \citeasnoun{mcleish_invariance_1975}
shows that (in this paper's notation) $\{P_T Z_{Ti}^2/b_T\}$ is
uniformly integrable,\footnote{Also see the remarks after
  \possessivecite{davidson_central_1992} Lemma 3.2.} so it is
sufficient to prove that  
\[
\outsum \big[h_c(Z_{Ti})^2 - \meR(h_c(Z_{Ti})^2 \mid \FT{R_T})\big]
\to 0
\]
in $L_1$ for any choice of $c$.

We prove this by showing that the array
\begin{equation} \label{eq_ap_array}
  \{h_c(Z_{Ti})^2 - \meR(h_c(Z_{Ti})^2 \mid \FT{R_T}), \FT{R_T +
    i\,b_T}\}
\end{equation}
 is another $L_2$-mixingale of size $-1/2$ with constants $d_{Ti}$
 that satisfy $\outsum d_{Ti}^2 \to 0$.  Fix $T$, $i$, and $\kappa$,
 and define the array $\{\tilde L_{jv}; v = R_T + (i-1)b_T + 1, \dotsc,
 R_T + i b_T; \, j = 1,2\}$ 
 independent of $\FT{R_T + (i-\kappa)b_T}$ using Lemma
 \ref{lem_coupling}.
 Now let
 \[
 W_{Ti} = \insum [\tilde L - \mmrT{}],
\]
so 
\begin{multline*}
\big\| \E\big[h_c(Z_{Ti})^2 - \meR(h_c(Z_{Ti})^2 \mid \FT{R_T}) \mid \FT{R_T +
  (i-\kappa)b_T}\big] \big\|_2 \\
= \big\| \E\big[ h_c(Z_{Ti})^2 - h_c(W_{Ti})^2 \mid \FT{R_T +
  (i-\kappa)b_T}\big] \big\|_2 \leq 
\big\| h_c(Z_{Ti})^2 - h_c(W_{Ti})^2 \big\|_2
\end{multline*}
and it suffices to bound the last quantity.

As in de Jong, we have the inequalities:
\begin{align*}
  \|h_c(Z_{Ti})^2 - h_c(W_{Ti})^2\|_2 
  & \leq 2 c \sqrt{b_T/P_T} \| h_c(Z_{Ti}) - h_c(W_{Ti}) \|_2 \\
  & \leq 2 c \sqrt{b_T/P_T} \; \Big\| 
    \sum_{s=R_T + (i-1)b_T + 1}^{R_T + i b_T} (\fT{}{s} - \tilde
  L_s)\Big\|_2 \\
  & \leq \left[2 c b_T^{3/2} P_T^{-1} \beta_{b_T}^{(\rho -
  2)/2\rho}\right] \beta_\kappa^{(\rho - 2)/2\rho}\\
  &\equiv d_{Ti} \; \beta_\kappa^{(\rho - 2)/2\rho}.
\end{align*}
Since $\beta_\kappa^{(\rho - 2)/2\rho}$, the array (\ref{eq_ap_array})
is an $L_2$-mixingale and has size $-1/2$, $\outsum 
d_{Ti}^2 \to 0$.  Then \possessivecite{mcleish_maximal_1975} Theorem 1.6
gives
\[
  \Big\| \outsum \big[h_c(Z_{Ti})^2 - \meR(h_c(Z_{Ti})^2
  \mid\FT{R_T})\big] \Big\|_1 =
  O\Big(\outsum d_{Ti}^2\Big),
\]
to complete the proof.
\end{proof}

\begin{proof}[Proof of Lemma \ref{lem_mixingale}]
We'll prove that 
\[
  \| \E(\fT{i}{R_T+j} - \mmrT{j} \mid \FT{R_T+j-\tau}) \|_2 
  \leq 2^{1+1/\rho} \; B_L \; \beta_\tau^{(\rho-2)/2\rho}
\]
Notice that $\beta_\tau^{(\rho-2)/2\rho} =
O(\tau^{-1/2-\delta}$. Define 
$\tilde L_s$ as in Lemma \ref{lem_coupling} to be independent of
$\FT{s-\tau}$.  Then
\begin{align*} \label{eq_mixingale_bound_1}
  \| \E(\fT{i}{s} - \mmrT{i} \mid \FT{s-\tau}) \|_2 
  & = \| \E(\fT{i}{s} - \tilde L_{is} \mid \FT{s-\tau}) \|_2 \\
  & \leq \| \fT{}{s} - \tilde L_{is} \|_2 \\
  & \leq 2^{1+1/\rho} B_L \beta_\tau^{(\rho-2)/2\rho}
\end{align*}
by Lemma \ref{lem_coupling}.
\end{proof}

\begin{proof}[Proof of Lemma \ref{lem_mixingale_normality}.]
\renewcommand{\insum}{\sum_{s=R_T + (i-1)b_T + l_T + 1}^{R_T + i b_T}}
Without loss of generality, assume that $\fT{}{s}$ is a scalar. 
We will modify \possessivecite{de_jong_central_1997} Theorem 1 to
establish normality. The only part of de Jong's proof that needs to be
changed is the handling of the covariance matrix; here it is a random
element and in de Jong's theorem it is a constant.

Let $\{b_T\}$, $\{l_T\}$, and $\{m_T\}$ be sequences of positive
integers that satisfy $P_T \geq b_T \geq l_T+1$, $b_T \to \infty$,
$l_T \to \infty$, $\lfloor P_T/b_T \rfloor \to \infty$, and
$l_T/b_T \to 0$. Then, De Jong proves that (in our notation)
\[
P_T^{-1/2} \big[\elT{} - \mmrT{}\big] = \outsum Z_{Ti} + o_p(1)
\]
with 
\[
Z_{Ti} \equiv \insum \big[\fT{}{s} - \E(\fT{}{s} \mid \FT{R_T + (i-1)b_T})\big].
\]
The array $\{Z_{Ti}, \FT{R_T + ib_T}, i = 1,\dotsc,m_T\}$ is a martingale 
difference array by construction and it suffices to apply a Central
Limit Theorem to $\sum_i Z_{Ti}$.

We apply \possessivecite{hall_martingale_1980} Theorem 3.3 to complete
the proof.\footnote{The covariance matrix, $\Sigma_T$, is measurable
  in all of the sub-sigma-fields $\FT{s}$, so Hall and Heyde's nesting
condition is unnecessary.  See the remarks surrounding their Theorem
for more details.  This measurability also allows us to use a sequence
of covariance matrices that does not necessarily converge.}  De Jong's
condition (9) ensures that Hall and Heyde's (3.18) and (3.20) are
satisfied, so it remains to prove that 
\[
  \outsum Z_{Ti}^2 = \Sigma_T + o_p(1).
\]
This last step is an immediate consequence of Lemma
 \ref{lem_cov_matrix_appendix} and De Jong's Lemmas 3 and 4.
\end{proof}

\begin{proof}[Proof of Theorem \ref{thm_nesting1}]
Under the null hypothesis,
\begin{align*}
  \elT{1} - \elT{2} &= [\elT{1} - \mmrT{1}] - [\elT{2} - \mmrT{2}]
  \\ &\quad + [\mmrT{1} - \mmmT{1}] + [\mmmT{2} - \mmrT{2}] \\
  & \leq [\elT{1} - \mmrT{1}] - [\elT{2} - \mmrT{2}] + o_p(R_T^{-1/2})
\end{align*}
since the derivative of $\mmT{1}{\cdot}$ at $\BT{1}$ is zero and
$[\mmmT{2} - \mmrT{2}]$ is positive.  Lemma
\ref{lem_mixingale_normality} ensures that this last quantity is
asymptotically normal with asymptotic variance $\hat \sigma_T$.
\end{proof}

\begin{proof}[Proof of Lemma \ref{lem_pd_conditions}]
  Let $\{(v_T, z_{1T}, z_{2T})\}$ be a sequence of random vectors,
  independent of $\bT{}{R_T}$ and equal in distribution to
  $\{(\epT{1}{t},\xT{1}{t},\xT{2}{t})\}$.  The prediction errors satisfy
  \begin{align*}
    \eT{1}{t} &= \epT{1}{t} + \xT{1}{t}'(\bT{1}{R_T} - \BT{1}) 
    \\ & = \epT{1}{t} + O_p(R_T^{-1/2}) \\
    \eT{2}{t} &= \epT{1}{t} + \xT{2}{t}'(\bT{2}{R_T} - \BT{2}).
  \end{align*}
  Since $z_T'(\bT{2}{R_T}- \BT{2})$ is a continuous random variable,
  the probability of it taking a value that guarantees constant loss
  is zero.
  To show that $[(1,-1) \SrT (1,-1)']^{-1} = O_p(1)$, then, it
  suffices to prove that the conditional variance (given $\bT{}{R_T}$)
  of the vector $(v_T^2,\, [v_T + z_T'(\bT{2}{R_T} -\BT{2})]^2)'$
  satisfies the same relationship. Since $z_T$ has uniformly positive
  variance, we only need to prove that $|\bT{2}{R_T} - \BT{2}|_2$ is
  uniformly a.s. positive.  This follows from the inequality
  \[
  |\bT{2}{R_T} - \BT{2}|_2^2 \geq
   \lambda_{max}(\XT{2}{R_T}'\XT{2}{R_T})^{-1}
   \lambda_{min}(\XT{2}{R_T}'\Ep{2}{R_T} \Ep{2}{R_T}'\XT{2}{R_T}).
  \]
\end{proof}

\begin{proof}[Proof of Lemma \ref{lem_muR}]
  Observe that 
  \begin{multline*}
    \mmrT{j} - \mmtT{j} = \\ \E\left[L(\psi_T - z_T'\bT{j}{R_T}) -
    L(\psi_T - z_T'\bT{j}{T}) \mid \bT{j}{R_T},\bT{j}{T}\right]
  \end{multline*}
  almost surely, with $(\psi_T, z_T) \eqdist (\yT{T+1},\xT{j}{T+1})$
  and independent of $(\bT{j}{R_T}, \bT{j}{T})$.  As a result, it
  suffices to show that
  \[
  \|L(\psi_T - z_T'\bT{j}{R_T}) - L(\psi_T - z_T'\bT{j}{T}) \|_1
  = O(\sqrt{P/R}).
  \]
  Since $L$ has finite left- and right-derivatives and is convex, 
  \[
  L(\psi_T - z_T'\bT{j}{R_T}) - L(\psi_T - z_T'\bT{j}{T})
  = O_p(1) z_T'(\bT{j}{R_T} - \bT{j}{T}),
  \]
  and, because this difference is uniformly integrable, we only need
  to prove that 
  \[
  |\bT{j}{R_T} - \bT{j}{T}|_2 = O_p(\sqrt{P_T/R_T}).
  \]

  Now, we can express this last difference as
  \begin{multline*}
  \bT{j}{R_T} - \bT{j}{T} = \left[
    (\XT{j}{T}'\XT{j}{T})^{-1} - (\XT{j}{R_T}'\XT{j}{R_T})^{-1}
  \right] \XT{j}{T}'\Ep{j}{T} \\
  + (\XT{j}{R_T}'\XT{j}{R_T})^{-1} \sum_{s=R_T+1}^T \xT{j}{s} \epT{j}{s}.
  \end{multline*}
  The square of each of these terms is $O_p(\sqrt{P_T/R_T})$.  First,
  observe that
  \begin{multline*}
    \left\lvert \left[
    (\XT{j}{T}'\XT{j}{T})^{-1} - (\XT{j}{R_T}'\XT{j}{R_T})^{-1}
  \right] \XT{j}{T}'\Ep{j}{T} \right\rvert_2^2\\
  = O_p(T) \sum_{i=1}^{K_{jT}} \lambda_i\left[(\XT{j}{T}'\XT{j}{T})^{-1}
    - (\XT{j}{R_T}'\XT{j}{R_T})^{-1}\right]^2,
  \end{multline*}
  which is $O_p(P_T/R_T)$ since 
  $(\XT{j}{T}'\XT{j}{T})^{-1} - (\XT{j}{R_T}'\XT{j}{R_T})^{-1}$ has
  rank $P_T$ and its largest eigenvalue is $O_p(1/T)$.
  A similar argument proves that the second term is
  $O_p(\sqrt{P_T/R_T})$ as well, completing the proof.
\end{proof}
%%

\bibliography{zoterolibrary}
\bibliographystyle{econometrica}
\end{document}
